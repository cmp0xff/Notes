\documentclass[a4paper,11pt]{article}
%\documentclass[a4paper,11pt]{scrartcl}



% Fonts and languages

% Multilingual support
%\usepackage{polyglossia}

% more symbols
\usepackage{textcomp}

% AMS--related
\usepackage{amsmath,amssymb}

% ':=' as \coloneqq
\usepackage{mathtools}
% Physical bras and kets
\usepackage{braket}
% SI units
\usepackage{siunitx}
\sisetup{separate-uncertainty}

\usepackage{graphicx}
%\usepackage[colorinlistoftodos]{todonotes}


% Boxed equations. NEED TO BE LOADED BEFORE unicode-math!
\usepackage{empheq}
% Theorems
\usepackage{amsthm}

% Chemical elements
%\usepackage[version=4]{mhchem}


%\usepackage{fancybox}

\usepackage{enumitem} % \begin{enumerate}[label=\Alph*]


% Other formats

% Labelling equations according to sections
\numberwithin{equation}{section}


% Bibliography in the main text!!!

%\usepackage[
			%style=alphabetic,
%			backend=biber]{biblatex}
%\usepackage{hyperref}
\usepackage{nameref}

% Cross-references
% The \newtheorem commands have to come after the loading of {cleveref}.
% Additionally, the cleverref package has to be loaded after ntheorem or
% amsthm. cleverref has to be loaded after hyperref!
\usepackage{cleveref}
%\usepackage{nameref}%,thmtools}

% Plot
\usepackage{tikz}
\usetikzlibrary{decorations.pathmorphing}
\usetikzlibrary{calc}
% save compiled tikz plots; enable --shell-escape
%\usetikzlibrary{external}
%\tikzexternalize[prefix=./tikz/]

% Fontspec fot XeLaTeX
\usepackage{fontspec}
	% Unicode fonts
	\setmainfont{CMU Serif}
	\setsansfont{CMU Sans Serif}
	\setmonofont{CMU Typewriter Text}
	% declare a command \doulos to load the Doulos SIL font
	%\newfontfamily\brill{Brill}
	% now create a \textIPA{} command
	%\DeclareTextFontCommand{\textIPA}{\brill}
\usepackage{amsfonts}
\usepackage{unicode-math}
\usepackage{unicode-math}
	\setmathfont{Latin Modern Math} % default
	%\setmathfont[range=\mathalpha]{Asana Math}
	\setmathfont{Asana Math}[range={\mathbin}] %\mathord
	\setmathfont{STIX Math}[range={"02609}] % ☉
	\setmathfont{XITS Math}[range={"1D4B6-"1D4CF}] % Script, Latin, lowercase
	\setmathfont{Latin Modern Math}[range={"1D608-"1D63B}, sans-style=italic]
	\setmathfont{Latin Modern Math}[range={
		"00391-"003A9,
		"003B1-"003F5, 
		"1D6A8-"1D6E1},	% Bold Greek
		sans-style=upright]
	%\setmathfont{⟨font name⟩}[range=⟨unicode range⟩,⟨font features⟩]

\input{../preambles/unicode}

\setmainlanguage{english}
\setotherlanguages{german,greek,russian}

\input{../preambles/math-single}
\input{../preambles/math-brac}
\input{../preambles/math-thm}
\input{../preambles/phys-chem}

\setromanfont[Mapping=tex-text]{Linux Libertine O}
% \setsansfont[Mapping=tex-text]{DejaVu Sans}
% \setmonofont[Mapping=tex-text]{DejaVu Sans Mono}

\usepackage[%style=authoryear-icomp,
			backend=biber]{biblatex}
\addbibresource{./maxwell-proca.bib}

\title{Hamiltonian Dynamics of Maxwell--Proca Theory in $\rbr{d+1}$-dimensions}
\author{Yi-Fan Wang (王\ 一帆)}
%\date{}

\begin{document}
\maketitle


\section{Maxwell--Proca theory in flat space-time}

Consider a Maxwell--Proca theory with source
\begin{equation}
\Ld = -\frac{1}{4} F_{\mu\nu} F^{\mu\nu} - \frac{1}{2} m^2 A_\mu A^\mu
+ A_\mu J^\mu,
\end{equation}
where $m > 0$ corresponds to the Proca theory \cite[sec.\ 2.3]{Gitman1990}, and 
$m = 0$ the Maxwell theory \cite[sec.\ 3.3.3]{Rothe2010}, \cite[sec.\ 
2.4]{Gitman1990}. In $\rbr{3+1}$ dimensions, the electric and magnetic fields 
are
\begin{align}
-F_{0i} &= F_{i0} = E_i = \partial_i A_0 - \partial_0 A_i, \\
F_{ij} &= \epsilon_{ijk}B^k, \qquad
B^i = \frac{1}{2} \epsilon^{ijk} F_{ij}.
\end{align}

The action with velocity is
\begin{align}
 \sfun{S^\text{v}}{A, \Pi, V} \coloneqq \int\dd t\int\dd^d x
\rbr{\Ld^\text{v} + \Pi^\mu\rbr{\dot A_\mu-V_\mu}},
\end{align}
where the Lagrangian density with velocity reads
\begin{align}
\Ld^\text{v} = \frac{1}{2} \rbr{V_i - \partial_i A_0}^2 - \frac{1}{4} F_{ij}^2 
+ \frac{m^2}{2} \rbr{A_0^2 - A_i^2} + A_0 J^0 + A_i J^i.
\label{eq:lagrangian-density-with-velocity}
\end{align}
In $\rbr{3+1}$ dimensions, \cref{eq:lagrangian-density-with-velocity} can also 
be written as
\begin{align}
\Ld^\text{v} = \frac{1}{2} \rbr{\vec E^2 - \vec B^2}
+ \frac{m^2}{2} \rbr{\Phi^2 - \vec A^2} - \rho\Phi + \vec A\cdot\vec J.
\end{align}
On the velocity shell, the canonical momenta densities are
\begin{equation}
\Pi^0 \coloneqq \frpa{\Ld^\text{v}}{V_0} = 0,\qquad
\Pi^i \coloneqq \frpa{\Ld^\text{v}}{V_i} = V^i - \partial^i A_0.
\end{equation}
The fundamental Poisson brackets are
\begin{equation}
\sbr{\rfun{A_\mu}{\vec x_1}, \rfun{\Pi^\nu}{\vec x_2}}_\text{P} =
\delta_\mu{}^\nu\rfun{\delta^d}{\vec x_1 - \vec x_2}.
\end{equation}

Brining the $V_i$'s on shell, the primary action reads
\begin{align}
\sfun{S^\text{p}}{A, \Pi, V_0} = \int \dd t\int\dd^d x
\rbr{\mscrH^\text{p}+\Pi^\mu \dot A_\mu + \partial_i\rbr{\Pi^i A_0}},
\end{align}
in which the primary Hamiltonian is
\begin{align}
\mscrH^\text{p} &= \frac{1}{2} \rbr{\Pi^i}^2 + \frac{1}{4} F_{ij}^2 + 
\frac{m^2}{2} \rbr{-A_0^2 + A_i^2} - A_i J^i \nonumber \\
&\quad + V_0 \Pi^0 - A_0 \rbr{\partial_i \Pi^i + J^0},
\label{eq:primary-Hamil-general-MP}
\end{align}
and
\begin{equation}
\Phi_1 \coloneqq \Pi^0
\end{equation}
is the only primary constraint.

\subsection{Constraint algebra}
The Poisson bracket of $\Phi_1$ and $\mscrH^\text{p}$ is
\begin{align}
\sbr{\rfun{\Phi_1}{\vec{x}_1},\rfun{\mscrH^\text{p}}{\vec{x}_2}}_\text{P} &=
\sbr{\rbr{\Pi^0}_1,
-\frac{m^2}{2}A_0^2-A_0\rbr{\partial_i \Pi^i + J^0}}_\text{P}
\nonumber \\
&= \rbr{-m^2 A^0 + \partial_i \Pi^i + J^0}_2 
\rfun{\delta}{\vec{x}_1-\vec{x}_2}.
\end{align}
Integration with $\dd^d x_2$ yields the secondary constraint
\begin{equation}
\sbr{\Phi_1, H^\text{p}}_\text{P} = -m^2 A^0 + \partial_i \Pi^i + J^0 
\eqqcolon \Phi_2,
\label{eq:secondary-general-MP}
\end{equation}
so that
\begin{equation}
\sbr{\rfun{\Phi_1}{\vec{x}_1},\rfun{\Phi_2}{\vec{x}_2}}_\text{P} = 
-m^2\rfun{\delta}{\vec{x}_1-\vec{x}_2}.
\label{eq:Maxwell-Proca-Q}
\end{equation}
One may further compute
\begin{align}
&\quad\sbr{\rfun{\Phi_2}{\vec{x}_1}, 
\rfun{\mscrH^\text{p}}{\vec{x}_2}}_\text{P} \nonumber \\
&= \sbr{\rbr{\partial_i \Pi^i}_1,
\rbr{\frac{1}{4} F_{jk}^2 + \frac{m^2}{2} A_j^2 - A_j J^j}_2}_\text{P} 
+ \sbr{-m^2 \rbr{A^0}_1, \rbr{V_0 \Pi^0}_2}_\text{P},
\end{align}
in which
\begin{align}
\sbr{\rbr{\partial_i \Pi^i}_1, \rbr{\frac{1}{4} 
F_{jk}^2}_2}_\text{P} &=
\rbr{\partial_j A_k - \partial_k A_j}_2 \rbr{\partial_i}_1
\sbr{\rbr{\Pi^i}_1, \rbr{\partial^j A^k}_2}_\text{P} \nonumber \\
&= -\rbr{F^{ij} \partial_j}_2 \rbr{\partial_i}_1
\rfun{\delta}{\vec{x}_1-\vec{x}_2}.
\end{align}
The Poisson bracket can be evaluated as
\begin{align}
&\quad \sbr{\rfun{\Phi_2}{\vec{x}_1}, 
\rfun{\mscrH^\text{p}}{\vec{x}_2}}_\text{P} \\
&= \rbr{-\rbr{F^{ij} \partial_j + m^2 A^i + J^i}_2 \rbr{\partial_i}_1 + m^2
\rbr{V_0}_2}
\rfun{\delta}{\vec{x}_1-\vec{x}_2}.
\end{align}
Integration with $\dd^d x_2$ yields
\begin{equation}
\sbr{\Phi_2, H^\text{p}}_\text{P} = -\partial_i\rbr{m^2 A^i + J^i} + m^2 V_0.
\label{eq:persistence-Phi2}
\end{equation}

\subsection{Proca theory}

For Proca theory $m > 0$ , then the algorithm terminates, and one obtains a 
pure second-class system.
\begin{equation}
\mbfQ = \begin{pmatrix}0 & -m^2 \\ +m^2 & 0\end{pmatrix},
\qquad
\mbfQ^{-1} = \begin{pmatrix}0 & +m^{-2} \\ -m^{-2} & 0\end{pmatrix}.
\end{equation}
Dirac bracket
\begin{equation}
\begin{split}
&\phantom{}
\sbr{\rfun{f}{\vec{x}_1},\rfun{g}{\vec{x}_2}}_\text{D} =
\sbr{\rbr{f}_1, \rbr{g}_2}_\text{P} \\
+ \int \dd^d x_3
\Bigl(&- \sbr{\rbr{f}_1, \rbr{\Pi^0}_3}_\text{P}
\sbr{\rbr{m^{-2}\partial_i \Pi^i + A_0}_3, \rbr{g}_2}_\text{P} \\
&+ \sbr{\rbr{f}_1, \rbr{m^{-2}\partial_i \Pi^i + A_0}_3}_\text{P}
\sbr{\rbr{\Pi^0}_3, \rbr{g}_2}_\text{P} \Bigr).
\end{split}
\end{equation}
The fundamental Dirac brackets, which are different from Poisson brackets, are
\begin{equation}
\begin{split}
\sbr{\rfun{A_0}{\vec{x}_1},\rfun{A_i}{\vec{x}_2}}_\text{D} &=
m^{-2} \rbr{\partial_i}_1 \rfun{\delta}{\vec{x}_1 - \vec{x}_2}, \\
\sbr{\rfun{A_0}{\vec{x}_1},\rfun{\Pi^0}{\vec{x}_2}}_\text{D} &= 0.
\end{split}
\end{equation}

\subsubsection{Physical coordinates}

Introducing the regularising coordinates
\begin{alignat}{3}
\alpha_i &= A_i + m^{-2}\rbr{\partial_i \Pi^0 - J_i},&\qquad \beta^i &= \Pi^i; 
\\
\alpha_0 &= A_0 + m^{-2}\rbr{\partial_i \Pi^i - J_0},&\qquad \beta^0 &= \Pi^0.
\end{alignat}
It is easy\footnote{Really? Have I done it?} to show that
\begin{align}
\sbr{\rfun{\alpha_i}{\vec{x}_1},\rfun{\beta^j}{\vec{x}_2}}_\text{D} &= 
\delta^i{}_j\rfun{\delta}{\vec{x}_1,\vec{x}_2},\\
\sbr{\rfun{\alpha_i}{\vec{x}_1},\rfun{\alpha_j}{\vec{x}_2}}_\text{D} &= 0 =
\sbr{\rfun{\beta^i}{\vec{x}_1},\rfun{\beta^j}{\vec{x}_2}}_\text{D}.
\end{align}
Furthermore, one has
\begin{equation}
\mscrH^\text{p} = \mscrH^\text{phy}+\mscrH^\text{con}+\mscrH^\text{irr},
\end{equation}
where\footnote{This is to be re-calculated, since a boundary term has been 
split at the beginning.}
\begin{align}
\begin{split}
\mscrH^\text{phy} &= \frac{1}{2}\rbr{\beta^i}^2 + \frac{m^2}{2}\alpha_i^2 +
\frac{1}{4}\rbr{\partial_i \alpha_j - \partial_j \alpha_i}^2 + 
\frac{1}{2m^2}\rbr{\partial_i \beta^i}^2 \\
&\quad +\frac{1}{m^2} J^0 \partial_i \beta^i,
\end{split}\\
\mscrH^\text{con} &= -\frac{m^2}{2}\alpha_0^2 - \frac{1}{2m^2} 
\rbr{\partial_i \beta^0}^2,\\
\begin{split}
\mscrH^\text{irr} &= \partial_i\rbr{\alpha_0\beta^i - \beta^0\alpha_i + 
\frac{1}{m^2}\rbr{\beta^0\partial_i\beta^0 - 
\beta^i\partial_j\beta^j - J^0 \beta^i}} \\
&\quad + \frac{1}{2m^2}\rbr{\rbr{J^0}^2 - \rbr{J^i}^2}.
\end{split}
\end{align}
Further more,
\begin{equation}
\Phi_1 = \beta^0,\qquad\Phi_2 = m^2\alpha_0 \propto \alpha_0.
\end{equation}
Thus the $\rbr{\alpha_i, \beta^i}$ are regular pairs of canonical variables, 
whereas $\rbr{\alpha_0, \beta^0}$ are the singular variables as constraints. 
The canonical dynamics of the physical $\rbr{\alpha_i, \beta^i}$'s are 
determined by $\mscrH^\text{phy}$ as a regular system.

\subsection{Free Maxwell theory}

For Maxwell theory $m = 0$. The primary Hamiltonian in 
\cref{eq:primary-Hamil-general-MP} takes the form
\begin{equation}
\mscrH^\text{p} = \frac{1}{2} \rbr{\Pi^i}^2 + \frac{1}{4} F_{ij}^2 - A_i J^i
+V_0 \Pi^0 - A_0 \rbr{\partial_i \Pi^i + J^0},
\label{eq:primary-Hamil-Maxwell}
\end{equation}
the secondary constraint $\Phi_2$ in \cref{eq:secondary-general-MP} now 
reads
\begin{equation}
\Phi_2 = \partial_i \Pi^i + J^0.
\end{equation}
In $\rbr{3+1}$ dimensions, the first two terms in 
\cref{eq:primary-Hamil-Maxwell} reads
\begin{equation}
\frac{1}{2}\rbr{\Pi^i}^2 + \frac{1}{4}F_{ij}^2 =
\frac{1}{2}\rbr{\vec E^2 + \vec B^2}.
\end{equation}


Since the Poisson bracket of $\Phi_2$ and $H^\text{p}$
\begin{equation}
\sbr{\Phi_2, H^\text{p}}_\text{P} = -\partial_i J^i
\end{equation}
contains now no canonical variable, the algorithm terminates. Furthermore, 
the constraint algebra is commutative, hence the system is a purely 
first-class one.

Persistence condition on $\Phi_2$ requires
\begin{align}
\partial_i J^i = 0,
\end{align}
which is confusing. 

\subsubsection{Physical coordinates}


\printbibliography

\end{document}
