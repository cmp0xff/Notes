%\special{papersize=8.5in,11in}
%\setlength{\parindent}{0pt}
\documentclass[12pt]{article}


%%%%%%%%%%%%%%%%%%%%%%%%%%%%%%%%%%%%%%%\input{preambles}
%%%% Packages %%%%

% AMS--related
\usepackage{amssymb,amsmath,amsthm}

\usepackage{upgreek}
\usepackage{mathtools}

\usepackage{dsfont}
\usepackage{slashed}
\usepackage{cancel}
\usepackage{fullpage}
\setlength{\parskip}{1em}
\usepackage{graphicx}

\usepackage{mathrsfs}

\usepackage[font=small,labelfont=bf]{caption} %Font size of Figure Caption
\usepackage{float}

\usepackage{authblk} % multiple authors


\usepackage{braket}
\usepackage{color}
\usepackage{siunitx}
\usepackage{xcolor}
\usepackage{cancel}
\usepackage{systeme}
\usepackage{bm}% bold math symbol, \bm, \hm. for vectors

\usepackage{epstopdf}% use eps figures

\usepackage[style=numeric-comp,
			backend=biber,
			sortlocale=de_DE,
			natbib=true,
			isbn=false,
			url=false, 
			doi=false,
			eprint=true,
			hyperref=true,
			sorting=none]{biblatex}
%https://tex.stackexchange.com/questions/116088/sort-biblatex-bibliography-by-ap
%pearance-of-cites-in-the-document
\addbibresource{./eft.bib}
\usepackage{hyperref}
\hypersetup{
     colorlinks   = true,
     citecolor    =  red
}
%\usepackage[noadjust]{cite}
\usepackage{cleveref} 

\usepackage{amsmath,amssymb}

% ':=' as \coloneqq
\usepackage{mathtools}
% Physical bras and kets
\usepackage{braket}
% SI units
\usepackage{siunitx}
\sisetup{separate-uncertainty}

%\usepackage[colorinlistoftodos]{todonotes}

% Boxed equations. NEED TO BE LOADED BEFORE unicode-math!
\usepackage{empheq}
% Theorems


% Plot
\usepackage{tikz}
\usetikzlibrary{decorations.pathmorphing}
\usetikzlibrary{calc}
% save compiled tikz plots; enable --shell-escape
%\usetikzlibrary{external}
%\tikzexternalize[prefix=./tikz/]


\usepackage{subcaption}

%%%% Chemical--Mathematical--Physical Constants %%%%
\newcommand\mi{\mathrm{i}} % imaginary unit i
\newcommand\me{\mathrm{e}} % natural number e

\usepackage{upgreek}

\newcommand\pp{\uppi}

\newcommand\const{\mathrm{const}} % constant

%%%% Math Commands without Parameter %%%%

\newcommand\dif{\mathrm{d}}
\newcommand\Dif{\mathrm{D}}

\DeclareMathOperator{\arcsinh}{arcsinh}
\DeclareMathOperator{\arccosh}{arccosh}
\DeclareMathOperator{\arctanh}{arctanh}
\DeclareMathOperator{\arccoth}{arccoth}
\DeclareMathOperator{\arcctgh}{arcctgh}
\DeclareMathOperator{\arcsech}{arcsech}
\DeclareMathOperator{\arccsch}{arccsch}

\DeclareMathOperator{\BesselJ}{J}
\DeclareMathOperator{\BesselY}{Y}
\DeclareMathOperator{\BesselF}{F}
\DeclareMathOperator{\BesselG}{G}
\DeclareMathOperator{\BesselI}{I}
\DeclareMathOperator{\BesselK}{K}
\DeclareMathOperator{\BesselL}{L}

\DeclareMathOperator{\sgn}{sgn}
\DeclareMathOperator{\grad}{grad}
\DeclareMathOperator{\curl}{curl}
\DeclareMathOperator{\rot}{rot}
\DeclareMathOperator{\opdiv}{div}
\DeclareMathOperator{\opdeg}{deg}

\DeclareMathOperator{\sech}{sech}
\DeclareMathOperator{\csch}{csch}

\DeclareMathOperator{\diag}{diag}
\DeclareMathOperator{\tr}{tr}
\DeclareMathOperator{\Tr}{Tr}
\DeclareMathOperator{\rank}{rank}

\DeclareMathOperator{\ad}{ad}

\DeclareMathOperator{\expi}{expi}


%%%% Math Commands with Parameters %%%%

% Bracket-like
\newcommand{\rbr}[1]{{\left(#1\right)}}
\newcommand{\sbr}[1]{{\left[#1\right]}}
\newcommand{\cbr}[1]{{\left\{#1\right\}}}
\newcommand{\abr}[1]{{\left<#1\right>}}
\newcommand{\vbr}[1]{{\left|#1\right|}}
\newcommand{\dvbr}[1]{{\left\|#1\right\|}}
\newcommand{\fat}[2]{{\left.#1\right|_{#2}}}
\newcommand*\abs[1]{\left|#1\right|}% \abs{}, absolute value bracket

% Functions; note the space between the name and the bracket!
\newcommand{\rfun}[2]{{#1}\mathopen{}\left(#2\right)\mathclose{}}
\newcommand{\sfun}[2]{{#1}\mathopen{}\left[#2\right]\mathclose{}}
\newcommand{\cfun}[2]{{#1}\mathopen{}\left\{#2\right\}\mathclose{}}
\newcommand{\afun}[2]{{#1}\mathopen{}\left<#2\right>\mathclose{}}
\newcommand{\vfun}[2]{{#1}\mathopen{}\left|#2\right|\mathclose{}}
% Fraction-like
\newcommand{\frde}[2]{{\frac{\dif{#1}}{\dif{#2}}}}
\newcommand{\frDe}[2]{{\frac{\Dif{#1}}{\Dif{#2}}}}
\newcommand{\frpa}[2]{{\frac{\partial{#1}}{\partial{#2}}}}
%\newcommand{\frdva}[2]{{\frac{\dva{#1}}{\dva{#2}}}}

% overline-like marks
\newcommand{\ol}[1]{{\overline{{#1}}}}
\newcommand{\ul}[1]{{\underline{{#1}}}}
\newcommand{\tld}[1]{{\widetilde{{#1}}}}
\newcommand{\ora}[1]{{\overrightarrow{#1}}}
\newcommand{\ola}[1]{{\overleftarrow{#1}}}
\newcommand{\td}[1]{{\widetilde{#1}}}
\newcommand{\what}[1]{{\widehat{#1}}}
%\newcommand{\prm}{{\symbol{"2032}}}

%%%%%%%%%%%%%%%%%%%%%%%%%%%%%%%%%%%%%%%%%%%%%%%%

\DeclarePairedDelimiter\ceil{\lceil}{\rceil}
\DeclarePairedDelimiter\floor{\lfloor}{\rfloor}


\def\beq{\begin{equation}}
\def\eeq{\end{equation}}

\def\a{\alpha}
\def\kap{\varkappa}
\def\e{\varepsilon}

%%%%%%%%%%%%%%%%%%%%%%%%%%%%%%%%%%%%%%%%%%%%%%%%

\newcommand*\ov[1]{\overset{(5)}{#1}}
\newcommand*\ovn[1]{\overset{(n)}{#1}}

%%%%%%%%%%%%%%%%%%%%%%%%%%%%%%%%%roman number
\makeatletter
\newcommand*{\rom}[1]{\expandafter\@slowromancap\romannumeral #1@}
\makeatother
%%%%%%%%%%%%%%%%%%%%%%%%%%%%%%%%%


\begin{document}

\title{Euler-Heisenberg Effective Action} 


\date{}


%\author[1,4]{Alexander A. Andrianov\thanks{a.andrianov@spbu.ru} }
%\author[2]{Chen Lan\thanks{stlanchen@yandex.ru} }
%\author[1]{Oleg O. Novikov\thanks{o.novikov@spbu.ru}}
%\author[3]{Yi-Fan Wang\thanks{yfwang@thp.uni-koeln.de}}
%\author[5]{Zhongyi Zhang\thanks{zhongyi@th.physik.uni-bonn.de}}



%\affil[1]{Saint-Petersburg State University, St. Petersburg 198504, Russia}
%\affil[2]{ELI-ALPS, ELI-Hu NKft, Dugonics t\'er 13, Szeged 6720, Hungary}
%\affil[3]{Institut f\"ur Theoretische Physik, Universit\"at zu K\"oln,
%Z\"ulpicher Stra\ss e 77, 50937 K\"oln, Germany}
%\affil[4]{Institut de Ci\`encies del Cosmos (ICCUB), Universitat de Barcelona, 
%Spain}
%\affil[5]{Physikalisches Institut, Universit\"at Bonn, Nussallee 12, 53115 
%Bonn, Germany}


\maketitle

%\input{./sections/0abstract}
%\abstract{}

%\tableofcontents


\section{Spinor electrodynamics in flat space-time}

Maxwell Lagrangian
\begin{equation}
\sfun{S_\text{Maxwell}}{A_\mu} \coloneqq \int\dif^{d+1} x\,\rbr{
-\frac{1}{4} F^{\mu\nu} F_{\mu\nu} }
\end{equation}

Dirac Lagrangian \citep[sec.~11]{weinberg1995}
\begin{equation}
\sfun{S_\text{Dirac}}{\psi, \bar\psi} \coloneqq \int\dif^{d+1} x\,
\cbr{-\bar{\psi} \rbr{\gamma^\mu\partial_\mu + m}\psi}.
\end{equation}

Interaction term
\begin{equation}
\sfun{S_\text{IDM}}{A_\mu, \psi, \bar\psi} \coloneqq \int\dif^{d+1} 
x\,\rbr{-\bar{\psi}\gamma^\mu\mi e A_\mu\psi}.
\end{equation}

The total action for spinor electrodynamics reads
\begin{align}
\sfun{S_{1/2}}{A_\mu, \psi, \bar\psi} &\coloneqq 
S_\text{Dirac} + S_\text{IDM}+S_\text{Maxwell} \nonumber \\
&= \int\dif^{d+1}x\,\cbr{-\bar\psi\rbr{\gamma^\mu D_\mu+m}\psi
-\frac{1}{4}F^{\mu\nu}F_{\mu\nu}},
\end{align}
where
\begin{align}
D_\mu\psi \coloneqq \rbr{\partial_\mu+\mi e A_\mu}\psi.
\end{align}


Generating functional
\begin{equation}
\sfun{\mathcal{Z}}{j^\mu, \bar\eta, \eta} \coloneqq
\int\Dif A\,\Dif\psi\,\Dif\bar\psi\,\cfun{\exp}{\mi\rbr{S_{1/2}
+\int\dif^{d+1} x\,\rbr{j^\mu A_\mu + \bar\eta \psi+\bar\psi \eta}}}
\end{equation}

Effective action
\begin{equation}
\sfun{\mathcal{Z}}{j^\mu, 0, 0} \eqqcolon
\int\Dif A\,\cfun{\exp}{\mi\rbr{S_\text{Maxwell} + 
\sfun{\varGamma}{A_\mu}
+\int\dif^{d+1} x \,j^\mu A_\mu}}.
\end{equation}
In other words,
\begin{align}
\cfun{\exp}{\mi\sfun{\varGamma}{A_\mu}} &\equiv \int
\Dif\psi\,\Dif\bar\psi\,\cfun{\exp}{\mi\rbr{S_\text{Dirac}+S_\text{IDM}}}
\nonumber \\
&\equiv \int\Dif\psi\,\Dif\bar\psi\,\cfun{\exp}{\mi\int\dif^{d+1} x\,
\bar{\psi} \rbr{-\slashed{\partial}-\mi e \slashed{A} - m}\psi}
\nonumber \\
&= \int\Dif\psi\,\Dif\bar\psi\,\cfun{\exp}{\mi\int\dif^{d+1} x\,\dif^{d+1} y\,
\rfun{\bar\psi}{x} \rfun{M}{x, y} \rfun{\psi}{y}}
\nonumber \\
&= \mathcal{N}\sfun{\det}{-\mi \rfun{M}{x, y}},
\end{align}
where
\begin{equation}
\rfun{M}{x, y} \coloneqq \rbr{+\slashed{\partial}_y-\mi e 
\rfun{\slashed{A}}{y} - m}\rfun{\delta^{d+1}}{x-y}.
\end{equation}

\begin{align}
\sfun{\varGamma}{A_\mu} &\equiv
-\mi\rbr{\ln\mathcal{N} + \ln \sfun{\det}{-\mi M}} \nonumber \\
&= -\mi \rbr{\ln\mathcal{N} + \Tr \sfun{\ln}{-\mi \rfun{M}{x, y}}}.
\end{align}

Note that
\begin{align}
\Tr \sfun{\ln}{-\mi \rfun{M}{x, y}} &\equiv
\Tr \sfun{\ln}{-\mi \rfun{M^\intercal}{x, y}} \nonumber \\
&= \Tr \sfun{\ln}{\mi\rbr{ +\slashed{\partial}_y^\intercal -\mi e 
\rfun{\slashed{A}^\intercal}{y} - m}\rfun{\delta^{d+1}}{x-y}} \nonumber \\
&= \Tr \sfun{\ln}{\mi\rbr{ -\mathscr{C}\slashed{\partial}_y\mathscr{C}^{-1}
+\mi e \mathscr{C}\rfun{\slashed{A}}{y}\mathscr{C}^{-1}
-\mathscr{C}m\mathscr{C}^{-1}} \rfun{\delta^{d+1}}{x-y}} \nonumber \\
&= \cfun{\Tr}{\mathscr{C}\sfun{\ln}{\mi\rbr{-\slashed{\partial}_y
+\mi e \rfun{\slashed{A}}{y} -m} \rfun{\delta^{d+1}}{x-y}}\mathscr{C}^{-1}} 
\nonumber \\
&= \Tr \sfun{\ln}{\mi\rbr{-\slashed{\partial}_y
+\mi e \rfun{\slashed{A}}{y} -m} \rfun{\delta^{d+1}}{x-y}}.
\end{align}
where the transpose ${}^\intercal$ is taken in the spinor space. Therefore
\begin{equation}
\Tr \sfun{\ln}{-\mi \rfun{M}{x, y}} =
\frac{1}{2} \Tr \sfun{\ln}{\rfun{M_2}{x, y}},
\label{eq:Tr-M_2}
\end{equation}
where
\begin{align}
\rfun{M_2}{x, y} \coloneqq \mathcal{M}_y \rfun{\delta^{d+1}}{x-y},\quad
\mathcal{M}_y \coloneqq \rbr{\rbr{\slashed{\partial}_y-\mi e 
\rfun{\slashed{A}}{y}}^2 - m^2}.
\label{eq:M_2-def}
\end{align}
One may further simplify \cref{eq:M_2-def} by noting ($y$ suppressed)
\begin{align}
\rbr{\slashed{\partial} -\mi e \slashed{A}}^2 &=
\partial^2 - e^2 A_\mu^2 -\mi e \frac{1}{2}\rbr{\sbr{\gamma^\mu, \gamma^\nu}_- 
+ \sbr{\gamma^\mu, \gamma^\nu}_+}
\rbr{\partial_\mu A_\nu + A_\mu \partial_\nu + A_\nu \partial_\mu} 
\nonumber \\
&= \partial^2 - e^2 A_\mu^2 - \mi e \rbr{\partial_\mu A^\mu + 2A^\mu 
\partial_\mu - \mi \sigma^{\mu\nu} \partial_\mu A_\nu} \nonumber \\
&= \rbr{\partial_\mu - \mi e A_\mu}^2
- e \sigma^{\mu\nu} \partial_{[\mu} A_{\nu]} \nonumber \\
&= \rbr{\partial_\mu - \mi e A_\mu}^2 - \frac{e}{2} \sigma^{\mu\nu} F_{\mu\nu},
\end{align}
so that
\begin{align}
\mathcal{M} \equiv \rbr{\rbr{\partial_{\mu} - \mi e A_\mu}^2 - 
\frac{e}{2} \sigma^{\mu\nu} F_{\mu\nu} - m^2},
\label{eq:M_2-simp}
\end{align}
where $y$ is suppressed as well.

\subsection{Constant background field in $\rbr{3+1}$-dimensions}

\Cref{eq:Tr-M_2} can be solved exactly when $F_{\mu\nu}$ is constant throughout 
space-time. Take the case \citep{resetsky2012} where $\vec{E}\parallel\vec{B}$ 
and, without loss of generality, $\vec{B}\parallel\vec{z}$. One has
\begin{equation}
F_{30} \equiv -F_{03} = E_3 \eqqcolon E,\qquad
F_{12} \equiv -F_{21} = B_3 \eqqcolon B.
\label{eq:const-EM-strength}
\end{equation}
A Landau-like choice of four-potential \citep{landau1930}
\begin{equation}
A_{\mu} \coloneqq \begin{pmatrix} 0 & -Bx_2 & 0 & Ex_0 \end{pmatrix}
\end{equation}
can be applied which leads to \cref{eq:const-EM-strength}. In this choice the
matrix reduces to
\begin{equation}
\mathcal{M} = 
\end{equation}

\cite{heisenberg1936}

%Further using the trick
%\begin{equation}
%\rfun{\ln}{A+\mi 0^+} =
%-\int_0^{+\infty}\frac{\dif s}{s}\,\me^{\mi s \rbr{A+\mi 0^+}},
%\end{equation}
%one obtains
%\begin{align}
%\sfun{\varGamma}{A_\mu} &= \frac{\mi}{2}\rbr{-\Tr \ln M_2-\ln \mathcal{N}_2}
%\nonumber \\
%&= \frac{\mi}{2}\rbr{\Tr\int_0^{+\infty}\frac{\dif s}{s}\,
%\me^{\mi s \rbr{M_2 + \mi 0^+}}-\ln \mathcal{N}_2} \nonumber \\
%&=
%\end{align}


\section{Scalar electrodynamics in flat space-time}

Complex Klein--Gordon Lagrangian
\begin{align}
\sfun{S_\text{CKG}}{\phi,\phi^*}\coloneqq\int\dif^{d+1}x\,\cbr{-\eta^{\mu\nu}
\rbr{\partial_\mu\phi}^* \rbr{\partial_\nu\phi} - m^2\phi^*\phi}.
\end{align}
Interaction term
\begin{align}
\sfun{S_\text{ICKGM}}{A_\mu,\phi,\phi^*} \coloneqq \int\dif^{d+1}x\,
\eta^{\mu\nu}{}
\cbr{\mi e A_\mu \rbr{-\phi^*\partial_\nu\phi+\phi\partial_\nu\phi^*}
+e^2 A_\mu A_\nu \phi^* \phi}.
\end{align}
The total action for scalar electrodynamics reads
\begin{align}
\sfun{S_0}{A_\mu, \phi, \phi^*} &\coloneqq 
S_\text{CKG} + S_\text{ICKGM} + S_\text{Maxwell} \nonumber \\
&= \int\dif^{d+1}x\,\cbr{-\eta^{\mu\nu}
\rbr{D_\mu\phi}^* \rbr{D_\nu\phi} - m^2\phi^*\phi
-\frac{1}{4}F^{\mu\nu}F_{\mu\nu}},
\end{align}
where
\begin{align}
D_\mu\phi \coloneqq \rbr{\partial_\mu+\mi e A_\mu}\phi.
\end{align}

Generating functional
\begin{equation}
\sfun{\mathcal{Z}}{j^\mu, \bar J, J} \coloneqq
\int\Dif A\,\Dif\phi\,\Dif\phi^*\,\cfun{\exp}{\mi\rbr{S_0
+\int\dif^{d+1} x\,\rbr{j^\mu A_\mu + J^* \psi+\psi^* J}}}.
\end{equation}

Effective action
\begin{equation}
\sfun{\mathcal{Z}}{j^\mu, 0, 0} \eqqcolon
\int\Dif A\,\cfun{\exp}{\mi\rbr{S_\text{Maxwell} + 
\sfun{\varGamma}{A_\mu}
+\int\dif^{d+1} x \,j^\mu A_\mu}}.
\end{equation}
In other words,
\begin{align}
\cfun{\exp}{\mi\sfun{\varGamma}{A_\mu}} &\coloneqq \int\Dif\phi\,\Dif\phi^*\,
\cfun{\exp}{\mi\rbr{S_\text{CKG}+S_\text{ICKGM}}}
\nonumber \\
&\equiv \int\Dif\phi\,\Dif\phi^*\,
\cfun{\exp}{\mi\int\dif^{d+1} x
\rbr{-\eta^{\mu\nu}
\rbr{D_\mu\phi}^* \rbr{D_\nu\phi} - m^2\phi^*\phi}}
\nonumber \\
&= \int\Dif\phi\,\Dif\phi^*\,
\cfun{\exp}{\mi\int\dif^{d+1} x\,\dif^{d+1} y\,
\rfun{\phi^*}{x} \rfun{M}{x, y} \rfun{\phi}{y}}
%\nonumber \\
%&= \mathcal{N}\sfun{\det}{-\mi \rfun{M}{x, y}},
\end{align}
where
\begin{align}
\rfun{M}{x, y} \coloneqq
%\rbr{+\slashed{\partial}_y-\mi e 
%\rfun{\slashed{A}}{y} - m}\rfun{\delta^{d+1}}{x-y}.
\end{align}

%\begin{align}
%\sfun{\varGamma}{A_\mu} &\equiv
%-\mi\rbr{\ln\mathcal{N} + \ln \sfun{\det}{-\mi M}} \nonumber \\
%&= -\mi \rbr{\ln\mathcal{N} + \Tr \sfun{\ln}{-\mi \rfun{M}{x, y}}}.
%\end{align}



\cite{weisskopf1936}



%\section*{Acknowledgements}
%\addcontentsline{toc}{section}{\protect\numberline{}Acknowledgements}%

\appendix

\section{Notions and conventions}

$\eta_{\mu\nu} \coloneqq \rfun{\diag}{-, +, +, +}$

Pauli matrices
\begin{equation}
\sigma^1 \coloneqq \begin{pmatrix} 0 & 1 \\ 1 & 0 \end{pmatrix},\quad
\sigma^2 \coloneqq \begin{pmatrix} 0 & -\mi \\ +\mi & 0 \end{pmatrix},\quad
\sigma^3 \coloneqq \begin{pmatrix} +1 & 0 \\ 0 & -1 \end{pmatrix}.
\end{equation}

\citep[sec.~5]{weinberg1995}
\begin{equation}
\sbr{\gamma^\mu, \gamma^\nu}_+ \coloneqq 2\eta^{\mu\nu}
\end{equation}

\begin{equation}
\mathscr{J}^{\mu\nu} \coloneqq -\frac{\mi}{4}\sbr{\gamma^\mu, \gamma^\nu}_-
\end{equation}

\begin{equation}
\sigma^{\mu\nu} \coloneqq \frac{\mi}{2}\sbr{\gamma^\mu, \gamma^\nu}_-
\equiv -2 \mathscr{J}^{\mu\nu}.
\end{equation}

Choose the chiral representation
\begin{equation}
\gamma^\mu = -\mi
\begin{bmatrix}0 & \sigma^\mu \\ \bar\sigma^\mu & 0\end{bmatrix},
\end{equation}
where
\begin{equation}
\sigma^\mu \coloneqq \rbr{1_2, +\vec{\sigma}},\qquad
\bar\sigma^\mu \coloneqq \rbr{1_2, -\vec{\sigma}}.
\end{equation}

\begin{align}
\sigma^{\mu\nu} &\equiv -\frac{\mi}{2}
\begin{bmatrix}
\sigma^\mu\bar\sigma^\nu-\sigma^\nu\bar\sigma^\mu & 0 \\
0 & \bar\sigma^\mu\sigma^\nu-\bar\sigma^\nu\sigma^\mu
\end{bmatrix} \nonumber \\
&=
\begin{cases}
0 & \mu = 0, \nu = 0; \\
\mi \begin{bmatrix}-\sigma^j & 0 \\ 0 & +\sigma^j\end{bmatrix}
& \mu = 0, \nu = j; \\
\mi \begin{bmatrix}+\sigma^i & 0 \\ 0 & -\sigma^i\end{bmatrix}
& \mu = i, \nu = 0; \\
\begin{bmatrix}
-\epsilon^{ij}{}_k \sigma^k & 0
\\ 0 & +\epsilon^{ij}{}_k \sigma^k\end{bmatrix}
& \mu = i, \nu = j.
\end{cases}
\end{align}


\section{Fresnel functional integral}

\citep[ch.~10]{mosel2004}


\printbibliography


\end{document}
