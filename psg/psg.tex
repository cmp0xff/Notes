\documentclass[a4paper,11pt]{article}
%\documentclass[a4paper,11pt]{scrartcl}



% Fonts and languages

% Multilingual support
%\usepackage{polyglossia}

% more symbols
\usepackage{textcomp}

% AMS--related
\usepackage{amsmath,amssymb}

% ':=' as \coloneqq
\usepackage{mathtools}
% Physical bras and kets
\usepackage{braket}
% SI units
\usepackage{siunitx}
\sisetup{separate-uncertainty}

\usepackage{graphicx}
%\usepackage[colorinlistoftodos]{todonotes}


% Boxed equations. NEED TO BE LOADED BEFORE unicode-math!
\usepackage{empheq}
% Theorems
\usepackage{amsthm}

% Chemical elements
%\usepackage[version=4]{mhchem}


%\usepackage{fancybox}

\usepackage{enumitem} % \begin{enumerate}[label=\Alph*]


% Other formats

% Labelling equations according to sections
\numberwithin{equation}{section}


% Bibliography in the main text!!!

%\usepackage[
			%style=alphabetic,
%			backend=biber]{biblatex}
%\usepackage{hyperref}
\usepackage{nameref}

% Cross-references
% The \newtheorem commands have to come after the loading of {cleveref}.
% Additionally, the cleverref package has to be loaded after ntheorem or
% amsthm. cleverref has to be loaded after hyperref!
\usepackage{cleveref}
%\usepackage{nameref}%,thmtools}

% Plot
\usepackage{tikz}
\usetikzlibrary{decorations.pathmorphing}
\usetikzlibrary{calc}
% save compiled tikz plots; enable --shell-escape
%\usetikzlibrary{external}
%\tikzexternalize[prefix=./tikz/]

% Fontspec fot XeLaTeX
\usepackage{fontspec}
	% Unicode fonts
	\setmainfont{CMU Serif}
	\setsansfont{CMU Sans Serif}
	\setmonofont{CMU Typewriter Text}
	% declare a command \doulos to load the Doulos SIL font
	%\newfontfamily\brill{Brill}
	% now create a \textIPA{} command
	%\DeclareTextFontCommand{\textIPA}{\brill}
\usepackage{amsfonts}
\usepackage{unicode-math}
\usepackage{unicode-math}
	\setmathfont{Latin Modern Math} % default
	%\setmathfont[range=\mathalpha]{Asana Math}
	\setmathfont{Asana Math}[range={\mathbin}] %\mathord
	\setmathfont{STIX Math}[range={"02609}] % ☉
	\setmathfont{XITS Math}[range={"1D4B6-"1D4CF}] % Script, Latin, lowercase
	\setmathfont{Latin Modern Math}[range={"1D608-"1D63B}, sans-style=italic]
	\setmathfont{Latin Modern Math}[range={
		"00391-"003A9,
		"003B1-"003F5, 
		"1D6A8-"1D6E1},	% Bold Greek
		sans-style=upright]
	%\setmathfont{⟨font name⟩}[range=⟨unicode range⟩,⟨font features⟩]

\input{../preambles/unicode}

\setmainlanguage{english}
\setotherlanguages{german,greek,russian}

\input{../preambles/math-single}
\input{../preambles/math-brac}
\input{../preambles/math-thm}
\input{../preambles/phys-chem}

%\setromanfont[Mapping=tex-text]{Linux Libertine O}
% \setsansfont[Mapping=tex-text]{DejaVu Sans}
% \setmonofont[Mapping=tex-text]{DejaVu Sans Mono}

\usepackage[%style=authoryear-icomp,
			backend=biber]{biblatex}
\addbibresource{./psg.bib}

\title{Cheat Sheet for Pseudo-Riemannian Geometry}
\author{Yi-Fan Wang (王\ 一帆)}
%\date{}

\begin{document}
\maketitle

%%%%%%%%%%%%%%%%
\subsection{Levi-Civita connection}

%%%%%%%%%%%%%%%%
An affine connection $\nabla$ is called a \emph{Levi-Civita connection} if

\begin{itemize}
\item
it preserves the metric, i.e.\ $\nabla g = 0$.
\item
it is torsion-free, i.e.\ for any vector fields $X$ and 
$Y$ we have $\nabla_{X}Y − \nabla_{Y}X
= [X, Y]$, where $[X, Y]$ is the Lie bracket of the vector fields $X$ and 
$Y$.

Condition 1 above is sometimes referred to as compatibility with the 
metric, and condition 2 is sometimes called symmetry, c.f.\ Do Carmo's text.

If a Levi-Civita connection exists, it is uniquely determined. Using conditions 
1 and the symmetry of the metric tensor $g$ we find:

\begin{align}
&\quad\,
X \bigl(g(Y,Z)\bigr) + Y \bigl(g(Z,X)\bigr) - Z \bigl(g(Y,X)\bigr) 
\nonumber \\
&= \rfun{g}{\nabla_X Y + \nabla_Y X, Z} + g(\nabla_X Z - \nabla_Z X, Y) + 
g(\nabla_Y Z - \nabla_Z Y, X).
\end{align}
By condition 2, the right hand side is equal to
\begin{align}
2g(\nabla_X Y, Z) - g\bigl([X,Y], Z\bigr) + g\bigl([X,Z],Y\bigr) + 
g\bigl([Y,Z],X\bigr),
\end{align}
so we find the Koszul formula
\begin{align}
2 g(\nabla_X Y, Z) &= 
X \bigl(g(Y,Z)\bigr) + Y \bigl(g(Z,X)\bigr) - Z \bigl(g(X,Y)\bigr) 
\nonumber \\
&\quad\, +
g\bigl([X,Y],Z\bigr) - g\bigl([Y,Z], X\bigr) - g\bigl([X,Z], Y\bigr)
\nonumber \\
&= \BbbL_Y \rfun{g}{X,Z} + \rfun{\rbr{\dd Y^{\flat}}}{X, Z}.
\end{align}
Since $Z$ is arbitrary, this uniquely determines $\nabla_{X}Y$. Conversely, 
using the last line as a definition one shows that the expression so defined is 
a connection compatible with the metric, i.e.\ is a Levi-Civita connection.

\printbibliography

\end{document}
