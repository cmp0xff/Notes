\documentclass[a4paper,11pt]{article}
%\documentclass[a4paper,11pt]{scrartcl}



% Fonts and languages

% Multilingual support
%\usepackage{polyglossia}

% more symbols
\usepackage{textcomp}

% AMS--related
\usepackage{amsmath,amssymb}

% ':=' as \coloneqq
\usepackage{mathtools}
% Physical bras and kets
\usepackage{braket}
% SI units
\usepackage{siunitx}
\sisetup{separate-uncertainty}

\usepackage{graphicx}
%\usepackage[colorinlistoftodos]{todonotes}


% Boxed equations. NEED TO BE LOADED BEFORE unicode-math!
\usepackage{empheq}
% Theorems
\usepackage{amsthm}

% Chemical elements
%\usepackage[version=4]{mhchem}


%\usepackage{fancybox}

\usepackage{enumitem} % \begin{enumerate}[label=\Alph*]


% Other formats

% Labelling equations according to sections
\numberwithin{equation}{section}


% Bibliography in the main text!!!

%\usepackage[
			%style=alphabetic,
%			backend=biber]{biblatex}
%\usepackage{hyperref}
\usepackage{nameref}

% Cross-references
% The \newtheorem commands have to come after the loading of {cleveref}.
% Additionally, the cleverref package has to be loaded after ntheorem or
% amsthm. cleverref has to be loaded after hyperref!
\usepackage{cleveref}
%\usepackage{nameref}%,thmtools}

% Plot
\usepackage{tikz}
\usetikzlibrary{decorations.pathmorphing}
\usetikzlibrary{calc}
% save compiled tikz plots; enable --shell-escape
%\usetikzlibrary{external}
%\tikzexternalize[prefix=./tikz/]

% Fontspec fot XeLaTeX
\usepackage{fontspec}
	% Unicode fonts
	\setmainfont{CMU Serif}
	\setsansfont{CMU Sans Serif}
	\setmonofont{CMU Typewriter Text}
	% declare a command \doulos to load the Doulos SIL font
	%\newfontfamily\brill{Brill}
	% now create a \textIPA{} command
	%\DeclareTextFontCommand{\textIPA}{\brill}
\usepackage{amsfonts}
\usepackage{unicode-math}
\usepackage{unicode-math}
	\setmathfont{Latin Modern Math} % default
	%\setmathfont[range=\mathalpha]{Asana Math}
	\setmathfont{Asana Math}[range={\mathbin}] %\mathord
	\setmathfont{STIX Math}[range={"02609}] % ☉
	\setmathfont{XITS Math}[range={"1D4B6-"1D4CF}] % Script, Latin, lowercase
	\setmathfont{Latin Modern Math}[range={"1D608-"1D63B}, sans-style=italic]
	\setmathfont{Latin Modern Math}[range={
		"00391-"003A9,
		"003B1-"003F5, 
		"1D6A8-"1D6E1},	% Bold Greek
		sans-style=upright]
	%\setmathfont{⟨font name⟩}[range=⟨unicode range⟩,⟨font features⟩]

\input{../preambles/unicode}

\setmainlanguage{english}
\setotherlanguages{german,greek,russian}

\input{../preambles/math-single}
\input{../preambles/math-brac}
\input{../preambles/math-thm}
\input{../preambles/phys-chem}

%\setromanfont[Mapping=tex-text]{Linux Libertine O}
% \setsansfont[Mapping=tex-text]{DejaVu Sans}
% \setmonofont[Mapping=tex-text]{DejaVu Sans Mono}

\usepackage[%style=authoryear-icomp,
			backend=biber]{biblatex}
\addbibresource{./string.bib}

\title{Classical $p$-brane}
\author{Yi-Fan Wang (王\ 一帆)}
%\date{}

\begin{document}
\maketitle

%%%%%%%%%%%%%%%%%%%%%%%%%%%%%%%%
\section{Point particle: $0$-brane}
\label{sec:pnt-ptc}
%%%%%%%%%%%%%%%%%%%%%%%%%%%%%%%%

%%%%%%%%%%%%%%%%
\subsection{Linear action}
\label{sec:pnt-ptc-lin}
%%%%%%%%%%%%%%%%

\begin{align}
\sfun{S_1}{x^{\Mu}} \coloneqq -m \int_{\gamma} \dd s = 
-m \int_{\gamma} \dd\lambda\, \sqrt{-g_{\Mu\Nu}
	\dot{x}^{\Mu} \dot{x}^{\Nu} }.
\end{align}

%%%%%%%%%%%%%%%%
\subsection{Quadratic action}
\label{sec:pnt-ptc-qua}
%%%%%%%%%%%%%%%%

\begin{align}
\sfun{S_2}{e, x^{\Mu}} \coloneqq \frac{1}{2} \int_{\gamma} \dd\lambda\, e
\rbr{e^{-2} g_{\Mu\Nu} \dot{x}^{\Mu} \dot{x}^{\Nu} - m^2}.
\end{align}


%%%%%%%%%%%%%%%%%%%%%%%%%%%%%%%%
\section{Classical bosonic string: $1$-brane}
\label{sec:bos-str}
%%%%%%%%%%%%%%%%%%%%%%%%%%%%%%%%

\cite{Johnson2000} contains a 

%%%%%%%%%%%%%%%%
\subsection{Nambu--Goto action}
\label{sec:bos-str-nam}
%%%%%%%%%%%%%%%%

The action reads \cite{Nambu1970a,Goto1971}
\begin{align}
\sfun{S_{\text{NG}}}{X^{\Mu}} \coloneqq -T \int_{\Sigma} \dd A = 
-T \int_{\Sigma} \dd^2 \sigma\,\sqrt{-\tsup[2]{\psi}},
\end{align}
where
\begin{align}
\tsup[2]{\psi} \coloneqq \det \psi_{\alpha\beta} \equiv
\psi_{11}\psi_{22} - \psi_{12}\psi_{21}
\end{align}
is the metric determinant, $\alpha, \beta, \ldots = 1, 2$ are the world-sheet 
indices, $\rbr{\Sigma^\alpha}$ are the world-sheet coordinates, 
\begin{align}
\psi_{\alpha\beta} \coloneqq g_{\Mu\Nu} 
X^{\Mu}{}_{,\alpha} X^{\Nu}{}_{,\beta},
\end{align}
is the induced metric on the world-sheet, $\Rho, \Nu, \ldots = 0, 1, \ldots d$ 
are the target-space indices, $X^{\Mu} = \rfun{X^{\Mu}}{\Sigma^{\alpha}}$ are 
the world-sheet \cite{Susskind1970} coordinates. The immersion map 
$\rfun{X^{\Mu}}{\sigma^\alpha}$ are the dynamical variables.

The inverse metric can also be expressed in a closed form
\begin{align}
\psi^{\alpha\beta} &= \frac{1}{\rbr{2-1}!} 
	\tsup[1]{\epsilon}^{\alpha\gamma} \tsup[1]{\epsilon}^{\beta\delta} 
	\tsup[2]{\psi}^{-1} \psi_{\gamma\delta}
\nonumber \\
&=
	\tsup[1]{\epsilon}^{\alpha\gamma} \tsup[1]{\epsilon}^{\beta\delta} 
	\tsup[2]{\psi}^{-1} \psi_{\gamma\delta} =
	\epsilon^{\alpha\gamma} \epsilon^{\beta\delta} \psi_{\gamma\delta}
\\
&=
\tsup[1]{\epsilon}^{\alpha\gamma} \tsup[1]{\epsilon}^{\beta\delta} 
	\tsup[2]{\psi}^{-1} g_{\Rho\Sigma} 
	X^{\Rho}{}_{,\gamma} X^{\Sigma}{}_{,\delta} =
\epsilon^{\alpha\gamma} \epsilon^{\beta\delta} g_{\Rho\Sigma} 
	X^{\Rho}{}_{,\gamma} X^{\Sigma}{}_{,\delta}
\\
&\equiv \frac{1}{\tsup[2]{\psi}}
\begin{pmatrix}
\psi_{22} & -\psi_{12} \\ -\psi_{21} & \psi_{11}\psi
\end{pmatrix}^{\alpha\beta}.
\end{align}
Unfortunately this is not useful.

Variation of the induced metric determinant can be expressed in terms of that 
of the induced metric
\begin{align}
\dva \tsup[2]{\psi} = \tsup[2]{\psi} \psi^{\alpha\beta}\,\dva 
	\psi_{\alpha\beta}.
\end{align}
Variation of the induced metric in terms of the world-sheet coordinates reads 
\begin{align}
\dva \psi_{\alpha\beta} = X^{\Nu}{}_{,\alpha} \rbr{
	2 g_{\Nu\Lambda}\, \dva X^{\Lambda}{}_{,\beta} + 
	X^{\Rho}{}_{,\beta} g_{\Nu\Rho,\Lambda}\, \dva X^{\Lambda}}.
\end{align}

Variation of the area element reads
\begin{align}
\dva \sqrt{-\tsup[2]{\psi}} &=
\frac{1}{2} \sqrt{-\tsup[2]{\psi}}\, \psi^{\alpha\beta}\,
\dva \psi_{\alpha\beta}
\nonumber \\
&=
\frac{1}{2} \sqrt{-\tsup[2]{\psi}}\, \psi^{\alpha\beta}
X^{\Nu}{}_{,\alpha}\rbr{2 g_{\Nu\Lambda}\, \dva X^{\Lambda}{}_{,\beta} + 
	X^{\Rho}{}_{,\beta} g_{\Nu\Rho,\Lambda}\, \dva X^{\Lambda}}
\nonumber \\
&=
-\frac{1}{T} \cbr{ \rbr{\ldots}_\text{NG}^{\beta}{}_{,\beta} +
	\mscrL_{\text{NG}, X^{\Lambda}}\,\dva X^\Lambda},
\end{align}
where
\begin{align}
\frac{1}{T} \rbr{\ldots}_\text{NG}^{\beta} &\coloneqq 
-\sqrt{-\tsup[2]{\psi}}\, 
\psi^{\alpha\beta} X^{\Nu}{}_{,\alpha} g_{\Nu\Lambda}\,\dva X^{\Lambda},
\end{align}
is a boundary term,
\begin{align}
\frac{\mscrL_{\text{NG}, X^{\Lambda}}}{T\sqrt{-\tsup[2]{\psi}}} &=
	\square_\psi X^{\Mu} g_{\Mu\Lambda} + 
	\psi^{\alpha\beta} X^{\Nu}{}_{,\alpha}
		g_{\Nu\Lambda,\Rho} X^{\Rho}{}_{,\beta} -
	\frac{1}{2} \psi^{\alpha\beta} X^{\Nu}{}_{,\alpha} X^{\Rho}{}_{,\beta}
		g_{\Nu\Rho,\Lambda}
\nonumber \\
&=
\square X^{\Mu} g_{\Mu\Lambda} + 
\frac{1}{2} \psi^{\alpha\beta} X^{\Nu}{}_{,\alpha} X^{\Rho}{}_{,\beta}
	\rbr{-g_{\Nu\Rho,\Lambda} + g_{\Rho\Lambda,\Nu} + g_{\Lambda\Nu,\Rho}}
\nonumber \\
&=
g_{\Mu\Lambda} \rbr{\square_\psi X^{\Mu} + \Gamma^{\Mu}{}_{\Nu\Rho} 
\psi^{\alpha\beta} X^{\Nu}{}_{,\alpha} X^{\Rho}{}_{,\beta}}
\end{align}
gives the Euler--Lagrange derivative, in which
\begin{align}
\square_\psi X^{\Mu} &\coloneqq
\frac{1}{\sqrt{-\tsup[2]{\psi}}} \rbr{\sqrt{-\tsup[2]{\psi}}\, 
	\psi^{\alpha\beta} X^{\Mu}{}_{,\alpha}}_{,\beta}
\end{align}
is a d'Alembertian on the world-sheet with respect to $\psi_{\alpha\beta}$.


%%%%%%%%%%%%%%%%
\subsection{External symmetry}
\label{sec:bos-str-ext-sym}
%%%%%%%%%%%%%%%%

If $\xi^\Mu$ is a Killing vector, one can show
\begin{align}
	\rbr{\sqrt{-\tsup[2]{\psi}}\, \psi^{\alpha\beta}
		X^{\Mu}{}_{,\alpha} \xi_{\Mu}}_{,\beta} = 0.
\end{align}

Geometrically, this corresponds to
\begin{align}
	\opdiv \xi^{\tang} = 0.
\end{align}
Since one also has
\begin{align}
	\BbbL_X \omega = \rbr{\opdiv X} \omega,
\end{align}
where $\omega$ is the volume form on $\Sigma$, one has 
\begin{align}
	\BbbL_{\xi^{\tang}} \omega = 0,
\end{align}
i.e.\ the volume form is invariant under diffeomorphism generated by 
$\xi^{\tang}$.

%%%%%%%%%%%%%%%%
\subsection{Polyakov action}
\label{sec:bos-str-pol}
%%%%%%%%%%%%%%%%

The action reads \cite{Deser_1976,Brink_1976,Polyakov1981}
\begin{align}
\sfun{S_\text{P}}{h_{\alpha\beta}, X^{\Mu}} = -\frac{T}{2} \int_{\Sigma}
\dd^2 \sigma\, \sqrt{-\tsup[2]{h}}\, h^{\alpha\beta} \psi_{\alpha\beta}.
\end{align}
Variation of the integrand reads
\begin{align}
\dva \rbr{\sqrt{-\tsup[2]{h}}\, h^{\alpha\beta} \psi_{\alpha\beta}}
&= \dva_h \rbr{\sqrt{-\tsup[2]{h}}\, h^{\alpha\beta}}\, \psi_{\alpha\beta}
+ \rbr{\sqrt{-\tsup[2]{h}}\, h^{\alpha\beta}}\, \dva_X  \psi_{\alpha\beta}
\nonumber \\
&= - \frac{2}{T} \cbr{ \rbr{\ldots}_{\text{P}}^{\beta}{}_{,\beta} + 
\mscrL_{\text{P}, h^{\alpha\beta}}\, \dva h^{\alpha\beta} +
\mscrL_{\text{P}, X^{\Lambda}}\, \dva X^{\Lambda} },
\end{align}
where
\begin{align}
\frac{2}{T} \rbr{\ldots}_{\text{P}}^{\beta}{}_{,\beta} &\coloneqq
- 2\sqrt{-\tsup[2]{h}}\, 
h^{\alpha\beta} X^{\Nu}{}_{,\alpha} g_{\Nu\Lambda}\,\dva X^{\Lambda},
\end{align}
is a boundary term,
\begin{align}
\frac{2\mscrL_{\text{P}, h^{\alpha\beta}}}{T \sqrt{-\tsup[2]{h}}} &= 
\rbr{\frac{1}{2} h^{\gamma\delta} h_{\alpha\beta} -
	\delta^{\gamma}{}_{\alpha} \delta^{\delta}{}_{\beta}}
X^{\Mu}{}_{,\gamma} X^{\Nu}{}_{,\delta} g_{\Mu\Nu}
\end{align}
gives the Euler--Lagrange derivative with respect to $h^{\alpha\beta}$,
\begin{align}
\frac{2 \mscrL_{\text{P}, X^{\Lambda}}}{T\sqrt{-\tsup[2]{h}}}  &= 2
g_{\Mu\Lambda}\rbr{\square_h X^{\Mu} +
\Gamma^{\Mu}{}_{\Nu\Rho} X^{\Nu}{}_{,\alpha} X^{\Rho}{}_{,\beta} 
h^{\alpha\beta}}
\end{align}
gives the Euler--Lagrange derivative with respect to $X^{\Lambda}$, and
\begin{align}
\square_h X^{\Mu} &\coloneqq
\frac{1}{\sqrt{-\tsup[2]{h}}} \rbr{\sqrt{-\tsup[2]{h}}\, 
	h^{\alpha\beta} X^{\Mu}{}_{,\alpha}}_{,\beta}
\end{align}
is another d'Alembertian on the world-sheet with respect to $h_{\alpha\beta}$.



%%%%%%%%%%%%%%%%%%%%%%%%%%%%%%%%
\section{$p$-brane}
\label{sec:p-bra}
%%%%%%%%%%%%%%%%%%%%%%%%%%%%%%%%

\cite{Anciaux2010}

\printbibliography

\end{document}
