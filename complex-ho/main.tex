\documentclass[a4paper]{article}


\input{../preambles/preamble}
\setdefaultlanguage{english}
%\setotherlanguages{german,greek}

\input{../preambles/mathematics}
\input{../preambles/physics}

%\newcommand{\RomaN}[1]{%
%  \textup{\uppercase\expandafter{\romannumeral#1}}%
%}


%\addbibresource{main.bib}
\title{Complex Harmonic Oscillators and How to Squeeze Them}
\author{YiFan Wang}

\newcommand{\Poibr}[1]{\sbr{#1}_\text{P}}
\newcommand{\Combr}[1]{\sbr{#1}_\text{-}}

\newcommand{\tR}{\Re}
\newcommand{\tI}{\Im}
\newcommand{\tS}{\text{S}}
\newcommand{\tD}{\text{D}}


\begin{document}
\maketitle

%\begin{abstract}
%Your abstract.
%\end{abstract}

\tableofcontents

\section{Single complex oscillator}

\begin{nameddef}{Classical Hamiltonian in canonical coordinates}
Complex phase space
\begin{equation}
H = \frac{1}{2}\pi^+\pi^- + \frac{\Omega^2}{2}\phi^+\phi^-,
\end{equation}
The corresponding Poisson brackets read
\begin{equation}
\Poibr{\rfun{f}{\eta^C}, \rfun{g}{\eta^C}} = \sum_C \rbr{
\frpa{f}{\phi^C}\frpa{g}{\pi^C} - \frpa{f}{\pi^C}\frpa{g}{\phi^C}},
\end{equation}
where $C \in \cbr{+, -}$, $\eta^+ = \rbr{\eta^-}^*$ and $\eta \in \cbr{\pi, 
\phi}$, so that
\begin{equation}
\Poibr{\phi^{C_1}, \pi^{C_2}} = \delta^{C_1 C_2},\qquad
\Poibr{\phi^{C_1}, \phi^{C_2}} = \Poibr{\pi^{C_1}, \pi^{C_2}} = 0.
\end{equation}

Real phase space
\begin{equation}
\phi^C = \frac{1}{\sqrt{2}}\rbr{ \phi_\tR - C \ii \phi_\tI},\qquad
\pi^C = \frac{1}{\sqrt{2}}\rbr{ \pi_\tR + C \ii \pi_\tI}.
%\phi^* &= \frac{1}{\sqrt{2}}\rbr{\phi^\tR - \ii \phi^\tI},\qquad
%\pi^* &&= \frac{1}{\sqrt{2}}\rbr{\pi^\tR + \ii \pi^\tI}.
\end{equation}
Inverse transformation
\begin{equation}
\begin{aligned}
\phi_\tR &= \frac{1}{\sqrt{2}}\rbr{\phi^+ + \phi^-},
&\phi_\tI &= \frac{\ii}{\sqrt{2}}\rbr{\phi^+ - \phi^-}, \\
\pi_\tR &= \frac{1}{\sqrt{2}}\rbr{\pi^- + \pi^+},
&\pi_\tI &= \frac{\ii}{\sqrt{2}}\rbr{\pi^- - \pi^+}. \\
\end{aligned}
\end{equation}

One can verify
\begin{equation}
H = \sum_{F} \frac{1}{2} \pi_F^2 + \frac{\Omega^2}{2} \phi_F^2,
\end{equation}
where $F \in \cbr{\tR, \tI}$, and
\begin{equation}
\Poibr{\rfun{f}{\eta_F}, \rfun{g}{\eta_F}} = \sum_F
\frpa{f}{\phi_F}\frpa{g}{\pi_F} - \frpa{f}{\pi_F}\frpa{g}{\phi_F},
\end{equation}
so that
\begin{equation}
\Poibr{\phi_{F_1}, \pi_{F_2}} = \delta_{F_1 F_2},\qquad
\Poibr{\phi_{F_1}, \phi_{F_2}} = \Poibr{\pi_{F_1}, \pi_{F_2}} = 0.
\end{equation}
hold as well.
\end{nameddef} % Classical Hamiltonian in canonical coordinates

\begin{nameddef}{Ladder coordinates}
Ladder coordinates (ladder `numbers', later to be quantised) in complex phase
space
\begin{align}
a_\phi^C &= \frac{1}{\sqrt{2}} \rbr{\Omega^{+\frac{1}{2}}\phi^{C} - C \ii
\Omega^{-\frac{1}{2}} \pi^{-C}}, \\
a_\pi^C &= \frac{1}{\sqrt{2}} \rbr{\Omega^{+\frac{1}{2}}\phi^{-C} - C \ii
\Omega^{-\frac{1}{2}} \pi^{C}},
\end{align}
where $-- = +$, $-+ = -$.
Poisson brackets? Inverse transformation
\begin{equation}
\phi^- = \frac{\Omega^{-\frac{1}{2}}}{\sqrt{2}}\rbr{a_\pi^+ + a_\phi^-},\qquad
\pi^- = \frac{\ii\Omega^{+\frac{1}{2}}}{\sqrt{2}}\rbr{a_\phi^+ - a_\pi^-}.
\end{equation}


Ladder coordinates in real phase space
\begin{equation}
a_F^C = \frac{1}{\sqrt{2}} \rbr{\Omega^{+\frac{1}{2}}\phi_F - C \ii
\Omega^{-\frac{1}{2}} \pi_F}.
\end{equation}
Poisson brackets? Inverse transformation
\begin{equation}
\phi_F = \frac{\Omega^{-\frac{1}{2}}}{\sqrt{2}}\rbr{a_F^+ + a_F^-},\qquad
\pi_F = \frac{\ii\Omega^{+\frac{1}{2}}}{\sqrt{2}}\rbr{a_F^+ - a_F^-}.
\end{equation}


One can check that
\begin{equation}
a_\phi^C = \frac{1}{\sqrt{2}}\rbr{a_\tR^C - C \ii a_\tI^C},\qquad
a_\pi^C = \frac{1}{\sqrt{2}}\rbr{a_\tR^C + C \ii a_\tI^C}.
\end{equation}
%Inverse transformation
%\begin{equation}
%a_\phi^C = \frac{1}{\sqrt{2}}\rbr{a_\tR^C - C \ii a_\tI^C},\qquad
%a_\pi^C = \frac{1}{\sqrt{2}}\rbr{a_\tR^C + C \ii a_\tI^C}.
%\end{equation}


\end{nameddef} % Ladder coordinates



\begin{nameddef}{Quantisation}
Quantisation in complex canonical coordinates
\begin{equation}
f \mapsto \what{f};\qquad
\Poibr{f,g} \mapsto \Combr{\what{f},\what{g}} = \ii \what{\Poibr{f,g}}.
\end{equation}
All classical equations listed above can be immediately quantised, since no 
product of non-commuting operators appears.

%Quantum Hamiltonian in canonical coordinates
%\begin{align}
%\what{H} &= \frac{1}{2} \what{\pi}^ + \what{\pi}^- + 
%\frac{\Omega^2}{2} \what{\phi}^+ \what{\phi}^-
%\\
%&= \sum_{F} \frac{1}{2} \what{\pi}_F^2 + \frac{\Omega^2}{2} \what{\phi}_F^2.
%\end{align}

Commutators of the ladder operators
\begin{align}
\Combr{\what{a}_{\eta_1}^{-C_1}, \what{a}_{\eta_2}^{C_2}} &= 
\delta_{\eta_1 \eta_2} \delta^{C_1 C_2} \what{1}; \\
\Combr{\what{a}_{F_1}^{-C_1}, \what{a}_{F_2}^{C_2}}
&= \delta_{F_1 F_2} \delta^{C_1 C_2} \what{1}. \\
\end{align}

Number operators
\begin{equation}
\what{n}_\eta \coloneqq \what{a}_\eta^+ \what{a}_\eta^-,\qquad
\what{n}_F \coloneqq \what{a}_F^+ \what{a}_F^-.
\end{equation}


Angular momentum operator
\begin{align}
\what{L} &\coloneqq
\what{\phi}_\tR\what{\pi}_\tI - \what{\phi}_\tI\what{\pi}_\tR \nonumber \\
&= \ii\rbr{\what{a}_\tI^+\what{a}_\tR^- - \what{a}_\tR^+\what{a}_\tI^-}
\nonumber \\
&= \ii\rbr{\what{\phi}^-\what{\pi}^- - \what{\phi}^+\what{\pi}^+}
= \ii\rbr{\what{\pi}^-\what{\phi}^- - \what{\pi}^+\what{\phi}^+}
\nonumber \\
&= \what{n}_\pi - \what{n}_\phi.
\label{eq:angular-momentum}
\end{align}
$\what{L} = \what{L}^\dagger$.

\begin{align}
\what{n}_\phi &= \frac{1}{2}
\rbr{\Omega^{+\frac{1}{2}}\what{\phi}^+ - \ii\Omega^{-\frac{1}{2}}\what{\pi}^-}
\rbr{\Omega^{+\frac{1}{2}}\what{\phi}^- + \ii\Omega^{-\frac{1}{2}}\what{\pi}^+}
\nonumber \\
&= \frac{1}{2} \rbr{\Omega^{+1} \what{\phi}^+\what{\phi}^- + 
\ii\rbr{\what{\phi}^+\what{\pi}^+ - \what{\pi}^-\what{\phi}^-}
+ \Omega^{-1} \what{\pi}^-\what{\pi}^+} \nonumber \\
&= \Omega^{-1}\what{H} - \frac{1}{2}\rbr{1+\what{L}}.
\end{align}
Substituting \eqref{eq:angular-momentum} yields the quantum Hamiltonian
\begin{empheq}[box=\fbox]{equation}
\what{H} = \frac{\Omega}{2}\rbr{\what{n}_\phi + \what{n}_\pi + 1}.
\end{empheq} % equation
%Luckily,
%\begin{equation}
%\Combr{\what{L},\what{H}} = 0
%\end{equation}
%so that
%\begin{equation}
%\Omega \Combr{\what{n},\what{L}} = \Combr{\what{n},\what{H}} = 0
%\end{equation}
%as well.


%\begin{equation}
%H = \frac{1}{2}\what{\pi}^+\what{\pi}^- + \frac{\Omega^2}{2}\phi^+\phi^-,
%\end{equation}
\end{nameddef}

\begin{nameddef}{Wave function}
One may choose the 
\end{nameddef}


\section{Rotating}

\section{Cohering}

\section{Single--mode squeezing}

\section{Double--mode squeezing}

\




% Let's print the overall heading of the bibliography first:
%\printbibheading
%\printbibliography

\end{document}