\documentclass[a4paper]{article}


% Fonts and languages

% Multilingual support
%\usepackage{polyglossia}

% more symbols
\usepackage{textcomp}

% AMS--related
\usepackage{amsmath,amssymb}

% ':=' as \coloneqq
\usepackage{mathtools}
% Physical bras and kets
\usepackage{braket}
% SI units
\usepackage{siunitx}
\sisetup{separate-uncertainty}

\usepackage{graphicx}
%\usepackage[colorinlistoftodos]{todonotes}


% Boxed equations. NEED TO BE LOADED BEFORE unicode-math!
\usepackage{empheq}
% Theorems
\usepackage{amsthm}

% Chemical elements
%\usepackage[version=4]{mhchem}


%\usepackage{fancybox}

\usepackage{enumitem} % \begin{enumerate}[label=\Alph*]


% Other formats

% Labelling equations according to sections
\numberwithin{equation}{section}


% Bibliography in the main text!!!

%\usepackage[
			%style=alphabetic,
%			backend=biber]{biblatex}
%\usepackage{hyperref}
\usepackage{nameref}

% Cross-references
% The \newtheorem commands have to come after the loading of {cleveref}.
% Additionally, the cleverref package has to be loaded after ntheorem or
% amsthm. cleverref has to be loaded after hyperref!
\usepackage{cleveref}
%\usepackage{nameref}%,thmtools}

% Plot
\usepackage{tikz}
\usetikzlibrary{decorations.pathmorphing}
\usetikzlibrary{calc}
% save compiled tikz plots; enable --shell-escape
%\usetikzlibrary{external}
%\tikzexternalize[prefix=./tikz/]

% Fontspec fot XeLaTeX
\usepackage{fontspec}
	% Unicode fonts
	\setmainfont{CMU Serif}
	\setsansfont{CMU Sans Serif}
	\setmonofont{CMU Typewriter Text}
	% declare a command \doulos to load the Doulos SIL font
	%\newfontfamily\brill{Brill}
	% now create a \textIPA{} command
	%\DeclareTextFontCommand{\textIPA}{\brill}
\usepackage{amsfonts}
\usepackage{unicode-math}
\usepackage{unicode-math}
	\setmathfont{Latin Modern Math} % default
	%\setmathfont[range=\mathalpha]{Asana Math}
	\setmathfont{Asana Math}[range={\mathbin}] %\mathord
	\setmathfont{STIX Math}[range={"02609}] % ☉
	\setmathfont{XITS Math}[range={"1D4B6-"1D4CF}] % Script, Latin, lowercase
	\setmathfont{Latin Modern Math}[range={"1D608-"1D63B}, sans-style=italic]
	\setmathfont{Latin Modern Math}[range={
		"00391-"003A9,
		"003B1-"003F5, 
		"1D6A8-"1D6E1},	% Bold Greek
		sans-style=upright]
	%\setmathfont{⟨font name⟩}[range=⟨unicode range⟩,⟨font features⟩]

\setdefaultlanguage{english}
%\setotherlanguages{german,greek}

% Math symbols and user-defined extensions


% some unicode characters
% ≙ for equal with hat


% Mathematical constants
\newcommand{\ii}{{\Bbbi}}
\newcommand{\ee}{{\Bbbe}}
\newcommand{\pp}{{\Bbbpi}}

% Bracket-like
\newcommand{\rbr}[1]{{\left(#1\right)}}
\newcommand{\sbr}[1]{{\left[#1\right]}}
\newcommand{\cbr}[1]{{\left\{#1\right\}}}
\newcommand{\abr}[1]{{\left<#1\right>}}
\newcommand{\vbr}[1]{{\left|#1\right|}}
\newcommand{\dvbr}[1]{{\left\|#1\right\|}}
\newcommand{\fat}[2]{{\left.#1\right|_{#2}}}
% Functions; note the space between the name and the bracket!
\newcommand{\rfun}[2]{{#1}\mathopen{}\left(#2\right)\mathclose{}}
\newcommand{\sfun}[2]{{#1}\mathopen{}\left[#2\right]\mathclose{}}
\newcommand{\cfun}[2]{{#1}\mathopen{}\left\{#2\right\}\mathclose{}}
\newcommand{\afun}[2]{{#1}\mathopen{}\left<#2\right>\mathclose{}}
\newcommand{\vfun}[2]{{#1}\mathopen{}\left|#2\right|\mathclose{}}
% Differentials
\newcommand{\DD}{\BbbD}
\newcommand{\dd}{\Bbbd}
\newcommand{\dva}{\mupdelta} % no better way?!
% Fraction-like
\newcommand{\frde}[2]{{\frac{\dd{#1}}{\dd{#2}}}}
\newcommand{\frDe}[2]{{\frac{\DD{#1}}{\DD{#2}}}}
\newcommand{\frpa}[2]{{\frac{\partial{#1}}{\partial{#2}}}}
\newcommand{\frdva}[2]{{\frac{\dva{#1}}{\dva{#2}}}}
% Equal marks
\newcommand{\eeq}{{\overset{!}{=}}}
\newcommand{\lls}{{\overset{!}{<}}}
\newcommand{\ggt}{{\overset{!}{>}}}
\newcommand{\lle}{{\overset{!}{\le}}}
\newcommand{\gge}{{\overset{!}{\ge}}}
% overline-like marks
\newcommand{\ol}[1]{{\overline{{#1}}}}
\newcommand{\ul}[1]{{\underline{{#1}}}}
\newcommand{\tld}[1]{{\widetilde{{#1}}}}
\newcommand{\ora}[1]{{\overrightarrow{#1}}}
\newcommand{\ola}[1]{{\overleftarrow{#1}}}
\newcommand{\td}[1]{{\widetilde{#1}}}
\newcommand{\what}[1]{{\widehat{#1}}}
%\newcommand{\prm}{{\symbol{"2032}}}

% Math operators
% Why does \DeclareMathOperator not work?
\DeclareMathOperator{\sgn}{sgn}
\DeclareMathOperator{\grad}{grad}
\DeclareMathOperator{\curl}{curl}
\DeclareMathOperator{\rot}{rot}
\DeclareMathOperator{\opdiv}{div}
\DeclareMathOperator{\opdeg}{deg}

\DeclareMathOperator{\sech}{sech}
\DeclareMathOperator{\csch}{csch}

\DeclareMathOperator{\diag}{diag}
\DeclareMathOperator{\tr}{tr}

\DeclareMathOperator{\ad}{ad}

\DeclareMathOperator{\expi}{expi}

% Group and Algebras
\newcommand{\SO}{\mathsf{SO}\,}
\newcommand{\SU}{\mathsf{SU}\,}
\newcommand{\so}{\mathfrak{so}\,}
\newcommand{\su}{\mathfrak{su}\,}


% amsthm
\newcommand{\thistheoremname}{} % for generic stuff
% definition of new styles
\newtheoremstyle{varplain}% name
	{}{}%      Spaces above and below, empty = `usual value'
	{\itshape}% Body font
	{}%         Indent amount (empty = no indent, \parindent = para indent)
	{\bfseries}% Thm head font
	{}%        Punctuation after thm head
	{\newline}% Space after thm head: \newline = linebreak
	{{\normalfont\thmnumber{(#2)}}\thmname{ #1}{\normalfont\thmnote{ (#3)}}}
	%         Thm head spec
\newtheoremstyle{vardefinition}% name
	{}{}%      Spaces above and below, empty = `usual value'
	{\upshape}% Body font
	{}%         Indent amount (empty = no indent, \parindent = para indent)
	{\bfseries}% Def head font
	{}%        Punctuation after thm head
	{\newline}% Space after thm head: \newline = linebreak
	{{\normalfont\thmnumber{(#2)}}\thmname{ #1}{\normalfont\thmnote{ (#3)}}}
	%         Thm head spec
\newtheoremstyle{varremark}% name
	{}{}%      Spaces above and below, empty = `usual value'
	{\upshape}% Body font
	{}%         Indent amount (empty = no indent, \parindent = para indent)
	{\itshape}% Rem head font
	{}%        Punctuation after thm head
	{\newline}% Space after thm head: \newline = linebreak
	{{\normalfont\upshape\thmnumber{(#2)}}\thmname{ #1}{\normalfont\thmnote{ (#3)}}}
	%         Thm head spec
% Plain style
\theoremstyle{plain}% default
\newtheorem{thm}{Theorem}[section]
\newtheorem{lem}[thm]{Lemma}
\newtheorem{prop}[thm]{Proposition}
% Definition style
\theoremstyle{definition}
\newtheorem{defn}{Definition}[section]
\newtheorem{exmp}{Example}[section]
\newtheorem{ppty}{Property}[section]
% Remark style
\theoremstyle{remark}
\newtheorem*{rem}{Remark}
% variant plain style
\theoremstyle{varplain}
% for specifying a named theorem with numbering
\newtheorem{genericthm}{\thistheoremname}[section]
\newenvironment{namedthm}[1]
	{\renewcommand{\thistheoremname}{#1}%
		\begin{genericthm}}
	{\end{genericthm}}
% for specifying a named theorem without numbering
\newtheorem*{genericthm*}{\thistheoremname}
\newenvironment{namedthm*}[1]
	{\renewcommand{\thistheoremname}{#1}%
		\begin{genericthm*}}
	{\end{genericthm*}}
\newtheorem{unamedthm}[genericthm]{Theorem}
% variant definition style
\theoremstyle{vardefinition}
% for specifying a named definition with numbering
\newtheorem{genericdef}[genericthm]{\thistheoremname}
\newenvironment{nameddef}[1]
	{\renewcommand{\thistheoremname}{#1}%
		\begin{genericdef}}
	{\end{genericdef}}
% for specifying a named definition without numbering
\newtheorem*{genericdef*}{\thistheoremname}
\newenvironment{nameddef*}[1]
	{\renewcommand{\thistheoremname}{#1}%
		\begin{genericdef*}}
	{\end{genericdef*}}
\newtheorem{unameddef}[genericthm]{Definition}
% variant remark style
\theoremstyle{varremark}
% for specifying a named  with numbering
\newtheorem{genericrem}[genericthm]{\thistheoremname}
\newenvironment{namedrem}[1]
	{\renewcommand{\thistheoremname}{#1}%
		\begin{genericrem}}
	{\end{genericrem}}
% for specifying a name without numbering
\newtheorem*{genericrem*}{\thistheoremname}
\newenvironment{namedrem*}[1]
	{\renewcommand{\thistheoremname}{#1}%
		\begin{genericrem*}}
	{\end{genericrem*}}
\newtheorem{unamedrem}[genericthm]{Remark}


% cleveref
\crefname{lem}{lemma}{lemmas}
\Crefname{lem}{Lemma}{Lemmas}
\crefname{thm}{theorem}{theorems}
\Crefname{thm}{Theorem}{Theorems}
\crefname{defn}{definition}{definitions}
\Crefname{defn}{Definition}{Definitions}
\crefname{exmp}{example}{examples}
\Crefname{exmp}{Example}{Examples}
\crefname{namedthm}{}{}
\Crefname{namedthm}{}{}
% Physical constants
\newcommand{\lc}{\mitsansc} % speed of light
\newcommand{\bk}{\mitsansk} % Boltzmann's constant
\newcommand{\phs}{\hslash} % reduced Planck constant
\newcommand{\ph}{\Planckconst} % Planck constant
\newcommand{\hH}{\mitsansH} % Hubble constant H
\newcommand{\hh}{\mitsansh} % Hubble constant h

\newcommand{\plm}{m_\text{P}} % Planck mass
\newcommand{\pll}{l_\text{P}} % Planck length
\newcommand{\plt}{t_\text{P}} % Planck time

\newcommand{\nG}{\mitsansG} % Newton's constant
\newcommand{\aN}{\mitsansN} % Avogadro number
\newcommand{\ec}{\mitsanse} % unit electric charge

\newcommand{\gR}{\mitsansR} % gas constant

\newcommand{\apE}{\alpha_\text{E}} % EM fine struct const
\newcommand{\apG}{\alpha_\text{G}} % Grav fine struct const

% Common symbols
\newcommand{\Ld}{\mscrL} % Lagrangian density
\newcommand{\fp}{p_\text{F}} % Fermi momentum
\newcommand{\fE}{\mscrE_\text{F}} % Fermi energy

% Others
\newcommand{\rSch}{R_\text{S}} % Schwarzschild radius

\newcommand{\fHor}{{h^+}} % future horizon
\newcommand{\pHor}{{h^-}} % past horizon

% siunitx
% Astronomy
\DeclareSIUnit\parsec{pc}
\DeclareSIUnit\lightyear{ly}

%\newcommand{\RomaN}[1]{%
%  \textup{\uppercase\expandafter{\romannumeral#1}}%
%}


%\addbibresource{main.bib}
\title{Complex Harmonic Oscillators and How to Squeeze Them}
\author{YiFan Wang}

\newcommand{\Poibr}[1]{\sbr{#1}_\text{P}}
\newcommand{\Combr}[1]{\sbr{#1}_\text{-}}

\newcommand{\tR}{\Re}
\newcommand{\tI}{\Im}
\newcommand{\tS}{\text{S}}
\newcommand{\tD}{\text{D}}


\begin{document}
\maketitle

%\begin{abstract}
%Your abstract.
%\end{abstract}

\tableofcontents

\section{Single complex oscillator}

\begin{nameddef}{Classical Hamiltonian in canonical coordinates}
Complex phase space
\begin{equation}
H = \frac{1}{2}\pi^+\pi^- + \frac{\Omega^2}{2}\phi^+\phi^-,
\end{equation}
The corresponding Poisson brackets read
\begin{equation}
\Poibr{\rfun{f}{\eta^C}, \rfun{g}{\eta^C}} = \sum_C \rbr{
\frpa{f}{\phi^C}\frpa{g}{\pi^C} - \frpa{f}{\pi^C}\frpa{g}{\phi^C}},
\end{equation}
where $C \in \cbr{+, -}$, $\eta^+ = \rbr{\eta^-}^*$ and $\eta \in \cbr{\pi, 
\phi}$, so that
\begin{equation}
\Poibr{\phi^{C_1}, \pi^{C_2}} = \delta^{C_1 C_2},\qquad
\Poibr{\phi^{C_1}, \phi^{C_2}} = \Poibr{\pi^{C_1}, \pi^{C_2}} = 0.
\end{equation}

Real phase space
\begin{equation}
\phi^C = \frac{1}{\sqrt{2}}\rbr{ \phi_\tR - C \ii \phi_\tI},\qquad
\pi^C = \frac{1}{\sqrt{2}}\rbr{ \pi_\tR + C \ii \pi_\tI}.
%\phi^* &= \frac{1}{\sqrt{2}}\rbr{\phi^\tR - \ii \phi^\tI},\qquad
%\pi^* &&= \frac{1}{\sqrt{2}}\rbr{\pi^\tR + \ii \pi^\tI}.
\end{equation}
Inverse transformation
\begin{equation}
\begin{aligned}
\phi_\tR &= \frac{1}{\sqrt{2}}\rbr{\phi^+ + \phi^-},
&\phi_\tI &= \frac{\ii}{\sqrt{2}}\rbr{\phi^+ - \phi^-}, \\
\pi_\tR &= \frac{1}{\sqrt{2}}\rbr{\pi^- + \pi^+},
&\pi_\tI &= \frac{\ii}{\sqrt{2}}\rbr{\pi^- - \pi^+}. \\
\end{aligned}
\end{equation}

One can verify
\begin{equation}
H = \sum_{F} \frac{1}{2} \pi_F^2 + \frac{\Omega^2}{2} \phi_F^2,
\end{equation}
where $F \in \cbr{\tR, \tI}$, and
\begin{equation}
\Poibr{\rfun{f}{\eta_F}, \rfun{g}{\eta_F}} = \sum_F
\frpa{f}{\phi_F}\frpa{g}{\pi_F} - \frpa{f}{\pi_F}\frpa{g}{\phi_F},
\end{equation}
so that
\begin{equation}
\Poibr{\phi_{F_1}, \pi_{F_2}} = \delta_{F_1 F_2},\qquad
\Poibr{\phi_{F_1}, \phi_{F_2}} = \Poibr{\pi_{F_1}, \pi_{F_2}} = 0.
\end{equation}
hold as well.
\end{nameddef} % Classical Hamiltonian in canonical coordinates

\begin{nameddef}{Ladder coordinates}
Ladder coordinates (ladder `numbers', later to be quantised) in complex phase
space
\begin{align}
a_\phi^C &= \frac{1}{\sqrt{2}} \rbr{\Omega^{+\frac{1}{2}}\phi^{C} - C \ii
\Omega^{-\frac{1}{2}} \pi^{-C}}, \\
a_\pi^C &= \frac{1}{\sqrt{2}} \rbr{\Omega^{+\frac{1}{2}}\phi^{-C} - C \ii
\Omega^{-\frac{1}{2}} \pi^{C}},
\end{align}
where $-- = +$, $-+ = -$.
Poisson brackets? Inverse transformation
\begin{equation}
\phi^- = \frac{\Omega^{-\frac{1}{2}}}{\sqrt{2}}\rbr{a_\pi^+ + a_\phi^-},\qquad
\pi^- = \frac{\ii\Omega^{+\frac{1}{2}}}{\sqrt{2}}\rbr{a_\phi^+ - a_\pi^-}.
\end{equation}


Ladder coordinates in real phase space
\begin{equation}
a_F^C = \frac{1}{\sqrt{2}} \rbr{\Omega^{+\frac{1}{2}}\phi_F - C \ii
\Omega^{-\frac{1}{2}} \pi_F}.
\end{equation}
Poisson brackets? Inverse transformation
\begin{equation}
\phi_F = \frac{\Omega^{-\frac{1}{2}}}{\sqrt{2}}\rbr{a_F^+ + a_F^-},\qquad
\pi_F = \frac{\ii\Omega^{+\frac{1}{2}}}{\sqrt{2}}\rbr{a_F^+ - a_F^-}.
\end{equation}


One can check that
\begin{equation}
a_\phi^C = \frac{1}{\sqrt{2}}\rbr{a_\tR^C - C \ii a_\tI^C},\qquad
a_\pi^C = \frac{1}{\sqrt{2}}\rbr{a_\tR^C + C \ii a_\tI^C}.
\end{equation}
%Inverse transformation
%\begin{equation}
%a_\phi^C = \frac{1}{\sqrt{2}}\rbr{a_\tR^C - C \ii a_\tI^C},\qquad
%a_\pi^C = \frac{1}{\sqrt{2}}\rbr{a_\tR^C + C \ii a_\tI^C}.
%\end{equation}


\end{nameddef} % Ladder coordinates



\begin{nameddef}{Quantisation}
Quantisation in complex canonical coordinates
\begin{equation}
f \mapsto \what{f};\qquad
\Poibr{f,g} \mapsto \Combr{\what{f},\what{g}} = \ii \what{\Poibr{f,g}}.
\end{equation}
All classical equations listed above can be immediately quantised, since no 
product of non-commuting operators appears.

%Quantum Hamiltonian in canonical coordinates
%\begin{align}
%\what{H} &= \frac{1}{2} \what{\pi}^ + \what{\pi}^- + 
%\frac{\Omega^2}{2} \what{\phi}^+ \what{\phi}^-
%\\
%&= \sum_{F} \frac{1}{2} \what{\pi}_F^2 + \frac{\Omega^2}{2} \what{\phi}_F^2.
%\end{align}

Commutators of the ladder operators
\begin{align}
\Combr{\what{a}_{\eta_1}^{-C_1}, \what{a}_{\eta_2}^{C_2}} &= 
\delta_{\eta_1 \eta_2} \delta^{C_1 C_2} \what{1}; \\
\Combr{\what{a}_{F_1}^{-C_1}, \what{a}_{F_2}^{C_2}}
&= \delta_{F_1 F_2} \delta^{C_1 C_2} \what{1}. \\
\end{align}

Number operators
\begin{equation}
\what{n}_\eta \coloneqq \what{a}_\eta^+ \what{a}_\eta^-,\qquad
\what{n}_F \coloneqq \what{a}_F^+ \what{a}_F^-.
\end{equation}


Angular momentum operator
\begin{align}
\what{L} &\coloneqq
\what{\phi}_\tR\what{\pi}_\tI - \what{\phi}_\tI\what{\pi}_\tR \nonumber \\
&= \ii\rbr{\what{a}_\tI^+\what{a}_\tR^- - \what{a}_\tR^+\what{a}_\tI^-}
\nonumber \\
&= \ii\rbr{\what{\phi}^-\what{\pi}^- - \what{\phi}^+\what{\pi}^+}
= \ii\rbr{\what{\pi}^-\what{\phi}^- - \what{\pi}^+\what{\phi}^+}
\nonumber \\
&= \what{n}_\pi - \what{n}_\phi.
\label{eq:angular-momentum}
\end{align}
$\what{L} = \what{L}^\dagger$.

\begin{align}
\what{n}_\phi &= \frac{1}{2}
\rbr{\Omega^{+\frac{1}{2}}\what{\phi}^+ - \ii\Omega^{-\frac{1}{2}}\what{\pi}^-}
\rbr{\Omega^{+\frac{1}{2}}\what{\phi}^- + \ii\Omega^{-\frac{1}{2}}\what{\pi}^+}
\nonumber \\
&= \frac{1}{2} \rbr{\Omega^{+1} \what{\phi}^+\what{\phi}^- + 
\ii\rbr{\what{\phi}^+\what{\pi}^+ - \what{\pi}^-\what{\phi}^-}
+ \Omega^{-1} \what{\pi}^-\what{\pi}^+} \nonumber \\
&= \Omega^{-1}\what{H} - \frac{1}{2}\rbr{1+\what{L}}.
\end{align}
Substituting \eqref{eq:angular-momentum} yields the quantum Hamiltonian
\begin{empheq}[box=\fbox]{equation}
\what{H} = \frac{\Omega}{2}\rbr{\what{n}_\phi + \what{n}_\pi + 1}.
\end{empheq} % equation
%Luckily,
%\begin{equation}
%\Combr{\what{L},\what{H}} = 0
%\end{equation}
%so that
%\begin{equation}
%\Omega \Combr{\what{n},\what{L}} = \Combr{\what{n},\what{H}} = 0
%\end{equation}
%as well.


%\begin{equation}
%H = \frac{1}{2}\what{\pi}^+\what{\pi}^- + \frac{\Omega^2}{2}\phi^+\phi^-,
%\end{equation}
\end{nameddef}

\begin{nameddef}{Wave function}
One may choose the 
\end{nameddef}


\section{Rotating}

\section{Cohering}

\section{Single--mode squeezing}

\section{Double--mode squeezing}

\




% Let's print the overall heading of the bibliography first:
%\printbibheading
%\printbibliography

\end{document}