%\special{papersize=8.5in,11in}
%\setlength{\parindent}{0pt}
\documentclass[12pt]{article}


%%%%%%%%%%%%%%%%%%%%%%%%%%%%%%%%%%%%%%%\input{preambles}
%%%% Packages %%%%

% AMS--related
\usepackage{amssymb,amsmath,amsthm}

\usepackage{upgreek}
\usepackage{mathtools}

\usepackage{dsfont}
\usepackage{slashed}
\usepackage{cancel}
\usepackage{fullpage}
\setlength{\parskip}{1em}
\usepackage{graphicx}

\usepackage{mathrsfs}

\usepackage[font=small,labelfont=bf]{caption} %Font size of Figure Caption
\usepackage{float}

\usepackage{authblk} % multiple authors


\usepackage{braket}
\usepackage{color}
\usepackage{siunitx}
\usepackage{xcolor}
\usepackage{cancel}
\usepackage{systeme}
\usepackage{bm}% bold math symbol, \bm, \hm. for vectors

\usepackage{epstopdf}% use eps figures

\usepackage{natbib}
% \usepackage[style=numeric-comp,
% 			backend=biber,
% 			natbib=true,
% 			isbn=false,
% 			url=false, 
% 			doi=false,
% 			eprint=true,
% 			hyperref=true,
% 			sorting=none]{biblatex}
%https://tex.stackexchange.com/questions/116088/sort-biblatex-bibliography-by-ap
%pearance-of-cites-in-the-document
%\addbibresource{./eft.bib}

\usepackage{hyperref}
\hypersetup{
     colorlinks   = true,
     citecolor    =  red
}
%\usepackage[noadjust]{cite}
\usepackage{cleveref} 

\usepackage{amsmath,amssymb}

% ':=' as \coloneqq
\usepackage{mathtools}
% Physical bras and kets
\usepackage{braket}
% SI units
\usepackage{siunitx}
\sisetup{separate-uncertainty}

%\usepackage[colorinlistoftodos]{todonotes}

% Boxed equations. NEED TO BE LOADED BEFORE unicode-math!
\usepackage{empheq}
% Theorems


% Plot
\usepackage{tikz}
\usetikzlibrary{decorations.pathmorphing}
\usetikzlibrary{calc}
% save compiled tikz plots; enable --shell-escape
%\usetikzlibrary{external}
%\tikzexternalize[prefix=./tikz/]


\usepackage{subcaption}

%%%% Chemical--Mathematical--Physical Constants %%%%
\newcommand\mi{\mathrm{i}} % imaginary unit i
\newcommand\me{\mathrm{e}} % natural number e

\usepackage{upgreek}

\newcommand\pp{\uppi}

\newcommand\const{\mathrm{const}} % constant

%%%% Math Commands without Parameter %%%%

\newcommand\dif{\mathrm{d}}
\newcommand\Dif{\mathrm{D}}

\DeclareMathOperator{\arcsinh}{arcsinh}
\DeclareMathOperator{\arccosh}{arccosh}
\DeclareMathOperator{\arctanh}{arctanh}
\DeclareMathOperator{\arccoth}{arccoth}
\DeclareMathOperator{\arcctgh}{arcctgh}
\DeclareMathOperator{\arcsech}{arcsech}
\DeclareMathOperator{\arccsch}{arccsch}

\DeclareMathOperator{\BesselJ}{J}
\DeclareMathOperator{\BesselY}{Y}
\DeclareMathOperator{\BesselF}{F}
\DeclareMathOperator{\BesselG}{G}
\DeclareMathOperator{\BesselI}{I}
\DeclareMathOperator{\BesselK}{K}
\DeclareMathOperator{\BesselL}{L}

\DeclareMathOperator{\sgn}{sgn}
\DeclareMathOperator{\grad}{grad}
\DeclareMathOperator{\curl}{curl}
\DeclareMathOperator{\rot}{rot}
\DeclareMathOperator{\opdiv}{div}
\DeclareMathOperator{\opdeg}{deg}

\DeclareMathOperator{\sech}{sech}
\DeclareMathOperator{\csch}{csch}

\DeclareMathOperator{\diag}{diag}
\DeclareMathOperator{\tr}{tr}
\DeclareMathOperator{\Tr}{Tr}
\DeclareMathOperator{\rank}{rank}

\DeclareMathOperator{\ad}{ad}

\DeclareMathOperator{\expi}{expi}


%%%% Math Commands with Parameters %%%%

% Bracket-like
\newcommand{\rbr}[1]{{\left(#1\right)}}
\newcommand{\sbr}[1]{{\left[#1\right]}}
\newcommand{\cbr}[1]{{\left\{#1\right\}}}
\newcommand{\abr}[1]{{\left<#1\right>}}
\newcommand{\vbr}[1]{{\left|#1\right|}}
\newcommand{\dvbr}[1]{{\left\|#1\right\|}}
\newcommand{\fat}[2]{{\left.#1\right|_{#2}}}
\newcommand*\abs[1]{\left|#1\right|}% \abs{}, absolute value bracket

% Functions; note the space between the name and the bracket!
\newcommand{\rfun}[2]{{#1}\mathopen{}\left(#2\right)\mathclose{}}
\newcommand{\sfun}[2]{{#1}\mathopen{}\left[#2\right]\mathclose{}}
\newcommand{\cfun}[2]{{#1}\mathopen{}\left\{#2\right\}\mathclose{}}
\newcommand{\afun}[2]{{#1}\mathopen{}\left<#2\right>\mathclose{}}
\newcommand{\vfun}[2]{{#1}\mathopen{}\left|#2\right|\mathclose{}}
% Fraction-like
\newcommand{\frde}[2]{{\frac{\dif{#1}}{\dif{#2}}}}
\newcommand{\frDe}[2]{{\frac{\Dif{#1}}{\Dif{#2}}}}
\newcommand{\frpa}[2]{{\frac{\partial{#1}}{\partial{#2}}}}
%\newcommand{\frdva}[2]{{\frac{\dva{#1}}{\dva{#2}}}}

% overline-like marks
\newcommand{\ol}[1]{{\overline{{#1}}}}
\newcommand{\ul}[1]{{\underline{{#1}}}}
\newcommand{\tld}[1]{{\widetilde{{#1}}}}
\newcommand{\ora}[1]{{\overrightarrow{#1}}}
\newcommand{\ola}[1]{{\overleftarrow{#1}}}
\newcommand{\td}[1]{{\widetilde{#1}}}
\newcommand{\what}[1]{{\widehat{#1}}}
%\newcommand{\prm}{{\symbol{"2032}}}

%%%%%%%%%%%%%%%%%%%%%%%%%%%%%%%%%%%%%%%%%%%%%%%%

\DeclarePairedDelimiter\ceil{\lceil}{\rceil}
\DeclarePairedDelimiter\floor{\lfloor}{\rfloor}


\def\beq{\begin{equation}}
\def\eeq{\end{equation}}

\def\a{\alpha}
\def\kap{\varkappa}
\def\e{\varepsilon}

%%%%%%%%%%%%%%%%%%%%%%%%%%%%%%%%%%%%%%%%%%%%%%%%

\newcommand*\ov[1]{\overset{(5)}{#1}}
\newcommand*\ovn[1]{\overset{(n)}{#1}}

%%%%%%%%%%%%%%%%%%%%%%%%%%%%%%%%%roman number
\makeatletter
\newcommand*{\rom}[1]{\expandafter\@slowromancap\romannumeral #1@}
\makeatother
%%%%%%%%%%%%%%%%%%%%%%%%%%%%%%%%%


\begin{document}

\title{Effective action of scalar electrodynamics} 


\date{}


%\author[1,4]{Alexander A. Andrianov\thanks{a.andrianov@spbu.ru} }
%\author[2]{Chen Lan\thanks{stlanchen@yandex.ru} }
%\author[1]{Oleg O. Novikov\thanks{o.novikov@spbu.ru}}
\author[3]{Yi-Fan Wang\thanks{yfwang@thp.uni-koeln.de}}
%\author[5]{Zhongyi Zhang\thanks{zhongyi@th.physik.uni-bonn.de}}



%\affil[1]{Saint-Petersburg State University, St. Petersburg 198504, Russia}
%\affil[2]{ELI-ALPS, ELI-Hu NKft, Dugonics t\'er 13, Szeged 6720, Hungary}
\affil[3]{Institut f\"ur Theoretische Physik, Universit\"at zu K\"oln,
Z\"ulpicher Stra\ss e 77, 50937 K\"oln, Germany}
%\affil[4]{Institut de Ci\`encies del Cosmos (ICCUB), Universitat de Barcelona, 
%Spain}
%\affil[5]{Physikalisches Institut, Universit\"at Bonn, Nussallee 12, 53115 
%Bonn, Germany}


\maketitle

%\input{./sections/0abstract}
%\abstract{}

%\tableofcontents

%%%%%%%%%%%%%%%%%%%%%%%%%%%%%%%%
\section{Wick rotation}

%%%%%%%%%%%%%%%%%%%%%%%%%%%%%%%%

Complex Klein--Gordon action in flat space-time
\begin{align}
\sfun{S_\text{CKG}}{\phi,\phi^*}\coloneqq\int\dif^{d+1}x\,\cbr{-\eta^{\mu\nu}
\rbr{\partial_\mu\phi}^* \rbr{\partial_\nu\phi} - m^2\phi^*\phi}.
\end{align}
Interaction terms
\begin{align}
\sfun{S_\text{ICKGM}}{A_\mu,\phi,\phi^*} \coloneqq \int\dif^{d+1}x\,
\eta^{\mu\nu}{}
\cbr{\mi e A_\mu \rbr{-\phi^*\partial_\nu\phi+\phi\partial_\nu\phi^*}
+e^2 A_\mu A_\nu \phi^* \phi}.
\end{align}
The total action for scalar electrodynamics reads
\begin{align}
\sfun{S}{A_\mu, \phi, \phi^*} &\coloneqq 
S_\text{CKG} + S_\text{ICKGM} + S_\text{Maxwell} \nonumber \\
&= \int\dif^{d+1}x\,\cbr{-\rbr{\nabla_{\!\mu} \phi}^* \rbr{\nabla^\mu\phi} - 
m^2\phi^*\phi -\frac{1}{4}F^{\mu\nu}F_{\mu\nu}},
% \nonumber \\
% &= \int\dif^{d+1} x \cbr{\phi^*\rbr{-\nabla_\mu \nabla^\mu - m^2}\phi
% - \frac{1}{4} F^{\mu\nu} F_{\mu\nu} + \text{b.t.}},
\end{align}
where
\begin{align}
\nabla_\mu\phi \coloneqq \rbr{\partial_\mu+\mi e A_\mu}\phi.
%,\qquad\text{b.t.} = \partial_\mu \rbr{\phi^*\nabla^\mu \phi}.
\end{align}

Wick rotation
\begin{align}
x_\text{E}^4 = \mi x^0,\quad A_4 = -\mi A_0,
\end{align}
so that
\begin{align}
\partial_{x^0} = \mi \partial_{x_\text{E}^4},\quad
F_{0i} = \mi F_{4i}.
\end{align}

The Euclidean action reads
\begin{align}
\sfun{S_\text{E}}{A_I,\phi,\phi^*} = \int \dif^{D} x_\text{E}\,\rbr{
	\frac{1}{4} F_{IJ} F^{IJ} + \rbr{\nabla_{\!I}\phi}^* \rbr{\nabla^I \phi} 
	+ m^2 \phi^* \phi}.
\end{align}


%%%%%%%%%%%%%%%%%%%%%%%%%%%%%%%%
\section{Euclidean signature}
\label{sec:eucl}
%%%%%%%%%%%%%%%%%%%%%%%%%%%%%%%%

Working with the Euclidean signature is much easier than in the Lorentzian 
signature.

%%%%%%%%%%%%%%%%
\subsection{Effective action}
\label{ssec:eucl-effe}
%%%%%%%%%%%%%%%%

Generating functional
\begin{align}
\sfun{\mathcal{Z}_\text{E}}{j^I, J, J^*} &\coloneqq
\int\Dif A\,\Dif\phi^*\,\Dif\phi\,\cfun{\exp}{-\rbr{S_\text{E}
+\int\dif^{D} x\,\rbr{j^I A_I + J^* \phi+\phi^* J}}}.
\end{align}
The scalar fields are to be integrated out. The derivative term can be 
rearranged
\begin{align}
\rbr{\nabla_{\! I} \phi}^* \rbr{\nabla^{I} \phi} = \partial_I\rbr{\phi^* 
\nabla^{I}\phi} - \phi^* \nabla_{\! I} \nabla^{I} \phi.
\end{align}
Hence
\begin{align}
S_\text{E} = \int \dif^D x\, \frac{1}{4} F_{IJ} F^{IJ} +
	\int \dif^D x\,\dif^D y\,
	\rfun{\phi^*}{x} \rfun{M_\text{E}}{x,y} \rfun{\phi}{y},
\end{align}
where
\begin{align}
\rfun{M_\text{E}}{x,y} \coloneqq \rbr{-\nabla_{\! x^I} \nabla^{x^I} + m^2} 
\rfun{\delta^{(D)}}{x-y},
\end{align}
see \cite[ch.\ 6]{mosel2004} for details. Now the scalar field can formally be 
integrated
\begin{align}
\sfun{\exp}{-\sfun{\varGamma_\text{SE}}{J^*, J}} &\coloneqq \int 
	\Dif\phi^*\,\Dif\phi\, \cfun{\exp}{-\int\dif^{D} x\, \rbr{
	\rbr{\nabla_{\! I} \phi}^* \rbr{\nabla^{I} \phi} + m^2 \phi^* \phi 
	+ J^* \phi + \phi^* J}}
\nonumber \\
&= \frac{1}{\det \rfun{M_{\text{E}}}{x,y}}
	\cfun{\exp}{-\int\dif^D x\,\dif^D y\,
		\rfun{J^*}{x}\rfun{D_\text{E}}{x-y} \rfun{J}{y} },
\end{align}
where
\begin{align}
\rfun{D_\text{E}}{x-y} \coloneqq \rfun{M_\text{E}^{-1}}{x,y} = 
	\frac{1}{\rbr{2\pp}^{D/2}} \rbr{\frac{m}{x}}^{\frac{D}{2}-1}
	\rfun{\BesselK_{\frac{D}{2}-1}}{m\rbr{x-y}}
\end{align}
is the Euclidean Green's function, calculated by Chao-Ming Jian (Everett You's 
notes). (\textbf{Check!})

The effective action of the scalar fields reads
\begin{align}
\sfun{\varGamma_{\text{SE}}}{J^*, J} &= \sfun{\varGamma_{\text{SE}}}{0,0} +
	\int\dif^D x\,\dif^D y\,
		\rfun{J^*}{x}\rfun{D_{\text{E}}}{x-y} \rfun{J}{y},
\\
\sfun{\varGamma_{\text{SE}}}{0,0} &=
	-\sfun{\ln}{\rbr{\det M_\text{E}}^{-1}} = \tr\ln M_\text{E},
\end{align}
which traces back to \cite{heisenberg1936,weisskopf1936}. The Euclidean 
generating functional now reads
\begin{align}
\sfun{\mathcal{Z}_\text{E}}{j^I, J, J^*} &=
\cfun{\exp}{- \int\dif^D x\,\dif^D y\,
	\rfun{J^*}{x}\rfun{D_\text{E}}{x,y} \rfun{J}{y}}
\nonumber \\
&\quad\,\cdot
\int\Dif A\, \cfun{\exp}{-\int \dif^D x\,
\rbr{\frac{1}{4} F_{IJ} F^{IJ} + j^{I} A_{I}} - \tr\ln M_\text{E}}
\nonumber \\
&\eqqcolon
\cfun{\exp}{- \int\dif^D x\,\dif^D y\,
	\rfun{J^*}{x}\rfun{D_\text{E}}{x,y} \rfun{J}{y}}
\sfun{\mathcal{Z}_\text{AE}}{0}
		\abr{\rfun{\exp}{-\tr\ln M_\text{E}}}_{j^I},
\label{eq:eucl-effe-100}
\end{align}
where the average is defined as
\begin{align}
\abr{\mathcal{O}}_{j^I} &\coloneqq \sfun{\mathcal{Z}_\text{AE}^{-1}}{0}
\int\Dif A\, \mathcal{O}\,\cfun{\exp}{-\int \dif^D x\,
	\rbr{\frac{1}{4} F_{IJ} F^{IJ} + j^{I} A_{I}}},
\\
\sfun{\mathcal{Z}_\text{AE}}{j^I} &\coloneqq
\int\Dif A\, \cfun{\exp}{-\int \dif^D x\,
\rbr{\frac{1}{4} F_{IJ} F^{IJ} + j^{I} A_{I}}}.
\end{align}


%%%%%%%%%%%%%%%%
\subsection{World-line formalism}
\label{ssec:eucl-wlfm}
%%%%%%%%%%%%%%%%

In \cref{eq:eucl-effe-100}, $\tr\ln M_\text{E}$ is crucial. Using the Schwinger 
integral representation \cite{Schwinger1951} (up to normalisation)
\begin{align}
\ln \alpha = -\int_0^{+\infty} \frac{\dif s}{s}\,\me^{-\alpha s},
\end{align}
one has (up to normalisation)
\begin{align}
- \tr\ln M_\text{E} = \int_0^{+\infty} \frac{\dif T}{T}\,
	\rfun{\exp}{-\frac{m^2 T}{2M}}\,
	\tr \rfun{\exp}{-\frac{M_\text{E}}{2M}},
\end{align}
where $T$ has the dimension of time, and $M$ that of mass, which will both be 
eliminated later. Introduce the Hamiltonian of a non-relativistic point particle 
(\textbf{check sign!})
\begin{align}
H \coloneqq \frac{1}{2M} \rbr{P_I + e A_I}^2,
\end{align}
so that quantisation yields the following representation (\textbf{check sign!})
\begin{align}
\tr \rfun{\exp}{-\frac{M_\text{E}}{2M}} &= \int_{-\infty}^{+\infty}\dif x\,
\Braket{x | \me^{-\widehat{H} T} | x}
\\
&= \oint \Dif x\,\cfun{\exp}{-\int_{0}^{T}\dif T'\,
	\rbr{\frac{M}{2} \rbr{\frde{x^I}{T'}}^2 + \mi e A_I \frde{x^I}{T'}}}.
\label{eq:eucl-wlfm-050}
\end{align}
Rescaling $T' \eqqcolon \lambda T$ gives
\begin{align}
- \tr\ln M_\text{E} = \int_{0}^{+\infty} \frac{\dif T}{T}\,
	\rfun{\exp}{-\frac{m^2 T}{2M}}
	\oint \Dif x\,
		\rfun{\exp}{-\frac{M}{2T} \int_{0}^{1} \dif \lambda\, \dot{x}_I^2
			- \mi e \oint A_I\,\dif x^I}.
\label{eq:eucl-wlfm-100}
\end{align}

%%%%%%%%%%%%%%%%
\subsection{Euler--Heisenberg effective Lagrangian}
\label{ssec:eucl-ehel}
%%%%%%%%%%%%%%%%

If the instanton magnetic field in \cref{eq:eucl-wlfm-050} is constant, the 
path integral can be performed exactly \cite{Feynman1965}. The result is the 
Euler--Heisenberg effective Lagrangian.

It is difficult to obtain a classical solution for the motion of a point 
particle in a more generic magnetic field, e.g.\ \cite{Kondo1964}. Therefore 
the generalisation in this direction is limited.

%%%%%%%%%%%%%%%%
\subsection{World-line instanton approximations}
\label{ssec:eucl-wlia}
%%%%%%%%%%%%%%%%

In \cref{eq:eucl-wlfm-100}, the $T$ integral can be performed first. Using the 
integral expression and the asymptotic expansion for a modified Bessel function
\begin{align}
\rfun{\BesselK_0}{x} &= \frac{1}{2} \int_0^{+\infty} \frde{\dif t}{t}\,
	\rfun{\exp}{-t-\frac{x^2}{4t}}
\\
&\approx \sqrt{\frac{\pp}{2x}}\,\me^{-x}\qquad
	x \gg 1,
\end{align}
one has
\begin{align}
- \tr\ln M_\text{E} &= 2 \oint \Dif x\,
	\rfun{\BesselK_0}{m\sqrt{\int_0^1 \dif\lambda\, \dot{x}_I^2}}
		\rfun{\exp}{- \mi e \oint A_I\,\dif x^I}
\\
&\approx \sqrt{\frac{2\pp}{m}} \oint \Dif x\,
	\rbr{\int_0^1 \dif\lambda\, \dot{x}_I^2}^{-1/4}
	\rfun{\exp}{-m\sqrt{\int_0^1 \dif\lambda\, \dot{x}_I^2}
		- \mi e \oint A_I\,\dif x^I},
\label{eq::eucl-wlia-100}
\end{align}
where \cref{eq::eucl-wlia-100} works for
\begin{align}
m \sqrt{\int_0^1 \dif\lambda\, \dot{x}_I^2} \gg 1
\quad\text{or}\quad
\int_0^1 \dif\lambda\, \dot{x}_I^2 \gg m^{-2}.
\end{align}
This idea traces back to \cite{Affleck1982}


Another loop-based approximation:
\begin{align}
\abr{\rfun{\exp}{-\tr\ln M_\text{E}}}_{j^I} \approx
\rfun{\exp}{-\abr{\tr\ln M_\text{E}}_{j^I}}.
\end{align}



%%%%%%%%%%%%%%%%
\subsection{Application of instanton approximation}

%%%%%%%%%%%%%%%%

\cite{Dunne2005}




%%%%%%%%%%%%%%%%%%%%%%%%%%%%%%%%
\section{Flat space-time (Lorentzian signature)}

%%%%%%%%%%%%%%%%%%%%%%%%%%%%%%%%



Generating functional
\begin{equation}
\sfun{\mathcal{Z}}{j^\mu, J^*, J} \coloneqq
\int\Dif A\,\Dif\phi\,\Dif\phi^*\,\cfun{\exp}{\mi\rbr{S_0
+\int\dif^{d+1} x\,\rbr{j^\mu A_\mu + J^* \phi+\phi^* J}}}.
\end{equation}

Effective action
\begin{equation}
\sfun{\mathcal{Z}}{j^\mu, 0, 0} \eqqcolon
\int\Dif A\,\cfun{\exp}{\mi\rbr{S_\text{Maxwell} + 
\sfun{\varGamma_\text{W}}{A_\mu}
+\int\dif^{d+1} x \,j^\mu A_\mu}}.
\end{equation}
In other words,
\begin{align}
\cfun{\exp}{\mi\sfun{\varGamma_\text{W}}{A_\mu}} &\coloneqq 
\int\Dif\phi\,\Dif\phi^*\,
\cfun{\exp}{\mi\rbr{S_\text{CKG}+S_\text{ICKGM}}}
\nonumber \\
&\equiv  \int\Dif\phi\,\Dif\phi^*\,
\cfun{\exp}{\mi\int\dif^{d+1}x\,\cbr{-\rbr{\nabla_\mu\phi}^* 
\rbr{\nabla^\mu\phi} - m^2\phi^*\phi}}.
\label{eq:scalar-qed-to-be-manipulated}
%\nonumber \\
\end{align}
The integral in the exponent can be manipulated; only the first term is 
essential
\begin{align}
&\quad\int\dif^{d+1}x\,\rbr{-\rbr{\nabla_\mu\phi}^* \rbr{\nabla^\mu\phi}} 
\nonumber \\
&= \int\dif^{d+1}x\,\dif^{d+1}y\,\rbr{
-\rbr{\nabla_{x^\mu}\rfun{\phi}{x}}^* \rfun{\delta^{d+1}}{x-y}
\nabla^{y^\mu}\rfun{\phi}{y}},
\label{eq:scalar-qed-matrix-1}
\end{align}
where
\begin{align}
\rfun{\delta^{d+1}}{x-y} \nabla^{y^\mu}\rfun{\phi}{y}
&= \rfun{\delta^{d+1}}{x-y}
\cbr{\partial^{y^\mu}+\mi e \rfun{A^\mu}{y}}\rfun{\phi}{y} \nonumber \\
&= \cbr{-\rbr{\nabla^{y^\mu}}^*\rfun{\delta^{d+1}}{x-y}}\rfun{\phi}{y}
+\partial^{y^\mu}B,
\end{align}
in which
\begin{align}
B = \rfun{B}{x,y} \coloneqq \rfun{\delta^{d+1}}{x-y}\rfun{\phi}{y};
\end{align}
going back to \cref{eq:scalar-qed-matrix-1},
\begin{align}
&= \int\dif^{d+1}x\,\dif^{d+1}y\,\cbr{
-\cbr{\cbr{\partial_{x^\mu}+\mi e \rfun{A_\mu}{x}}\rfun{\phi}{x}}^* 
\rfun{\delta^{d+1}}{x-y} \nabla^{y^\mu}\rfun{\phi}{y}}
\nonumber \\
&= \int\dif^{d+1}x\,\dif^{d+1}y\,\cbr{ -\partial_{x^\mu} C^\mu + 
\rfun{\phi^*}{x}
\nabla_{x^\mu}
\rfun{\delta^{d+1}}{x-y} \nabla^{y^\mu}\rfun{\phi}{y}} \nonumber \\
&= \int\dif^{d+1}x\,\dif^{d+1}y\,\cbr{ -\partial_{x^\mu} C^\mu + 
\rfun{\phi^*}{x}
\cbr{-\rbr{\nabla_{y^\mu}}\rbr{\nabla^{y^\mu}}^*\rfun{\delta^{d+1}}{x-y}}
\rfun{\phi}{y} + \partial^{y^\mu}\nabla_{x^\mu} B},
\end{align}
in which
\begin{equation}
C^\mu = \rfun{C^\mu}{x,y} \coloneqq \rfun{\phi^*}{x} 
\rfun{\delta^{d+1}}{x-y} \nabla^{y^\mu}\rfun{\phi}{y}.
\end{equation}
Now \cref{eq:scalar-qed-to-be-manipulated} can be written as (dropping the 
boundary terms)
\begin{align}
&= \int\Dif\phi\,\Dif\phi^*\,
\cfun{\exp}{-\mi\int\dif^{d+1} x\,\dif^{d+1} y\,
\rfun{\phi^*}{x} \rfun{D^{-1}}{x, y} \rfun{\phi}{y}}
\nonumber \\
&= \tilde{\mathcal{N}}\cbr{\sfun{\det}{\rfun{D^{-1}}{x, y}}}^{-1/2},
\end{align}
where
\begin{align}
\rfun{D^{-1}}{x, y} \coloneqq \mathcal{D}^{-1}_y \rfun{\delta^{d+1}}{x-y},\qquad
\mathcal{D}^{-1}_y \coloneqq +\rbr{\nabla_{y^\mu}}\rbr{\nabla^{y^\mu}}^* + m^2.
\end{align}

\begin{align}
\sfun{\varGamma_\text{W}}{A_\mu} &\equiv
-\mi\rbr{\ln\tilde{\mathcal{N}} - \frac{1}{2} \ln \det D^{-1}} \nonumber \\
&= \frac{\mi}{2} \tr_x \rfun{\ln}{\mathcal{N}^{-1} D^{-1}} \nonumber \\
&= \frac{\mi}{2} \int_0^{+\infty}\frac{\dif s}{s}
\int\dif^{d+1}x\,\dif^{d+1}y\,\rfun{\delta^{d+1}}{x-y} \nonumber \\
&\quad\cdot\cbr{
-\me^{\mi s\rbr{+\rbr{\nabla_{y^\mu}}\rbr{\nabla^{y^\mu}}^* + m^2 + \mi 0^+}
\rfun{\delta^{d+1}}{x-y}}
+\me^{\mi s\rbr{\mathcal{N}+\mi 0^+}}}.
\end{align}

%\begin{align}
%\sfun{\varGamma_\text{W}}{A_\mu} &\equiv
%-\mi\rbr{\ln\mathcal{N} + \ln \sfun{\det}{-\mi M}} \nonumber \\
%&= -\mi \rbr{\ln\mathcal{N} + \Tr \sfun{\ln}{-\mi \rfun{D^{-1}}{x, y}}}.
%\end{align}



\cite{weisskopf1936}



%\section*{Acknowledgements}
%\addcontentsline{toc}{section}{\protect\numberline{}Acknowledgements}%

\appendix

%%%%%%%%%%%%%%%%%%%%%%%%%%%%%%%%
\section{Notions and conventions}

%%%%%%%%%%%%%%%%%%%%%%%%%%%%%%%%

The metric convention is mostly positive, i.e.\
$\eta_{\mu\nu} \coloneqq \rfun{\diag}{-, +, +, \ldots}$

Pauli matrices
\begin{equation}
\sigma^1 \coloneqq \begin{pmatrix} 0 & 1 \\ 1 & 0 \end{pmatrix},\quad
\sigma^2 \coloneqq \begin{pmatrix} 0 & -\mi \\ +\mi & 0 \end{pmatrix},\quad
\sigma^3 \coloneqq \begin{pmatrix} +1 & 0 \\ 0 & -1 \end{pmatrix}.
\end{equation}

The $\gamma$-matrices satisfy \cite[sec.~5]{weinberg1995}
\begin{equation}
\sbr{\gamma^\mu, \gamma^\nu}_+ \coloneqq 2\eta^{\mu\nu} \mathbf{1}_4.
\end{equation}

\begin{equation}
\mathscr{J}^{\mu\nu} \coloneqq -\frac{\mi}{4}\sbr{\gamma^\mu, \gamma^\nu}_-
\end{equation}

\begin{equation}
\sigma^{\mu\nu} \coloneqq \frac{\mi}{2}\sbr{\gamma^\mu, \gamma^\nu}_-
\equiv -2 \mathscr{J}^{\mu\nu}.
\end{equation}

In $\rbr{3+1}$ dimensions, choose the chiral representation
\begin{equation}
\gamma^\mu = -\mi
\begin{bmatrix}0 & \sigma^\mu \\ \bar\sigma^\mu & 0\end{bmatrix},
\end{equation}
where
\begin{equation}
\sigma^\mu \coloneqq \rbr{1_2, +\vec{\sigma}},\qquad
\bar\sigma^\mu \coloneqq \rbr{1_2, -\vec{\sigma}}.
\end{equation}

\begin{align}
\sigma^{\mu\nu} &\equiv -\frac{\mi}{2}
\begin{bmatrix}
\sigma^\mu\bar\sigma^\nu-\sigma^\nu\bar\sigma^\mu & 0 \\
0 & \bar\sigma^\mu\sigma^\nu-\bar\sigma^\nu\sigma^\mu
\end{bmatrix} \nonumber \\
&=
\begin{cases}
0 & \mu = 0, \nu = 0; \\
\mi \begin{bmatrix}+\sigma^j & 0 \\ 0 & -\sigma^j\end{bmatrix}
& \mu = 0, \nu = j; \\
\mi \begin{bmatrix}-\sigma^i & 0 \\ 0 & +\sigma^i\end{bmatrix}
& \mu = i, \nu = 0; \\
\begin{bmatrix}
+\epsilon^{ij}{}_k \sigma^k & 0
\\ 0 & +\epsilon^{ij}{}_k \sigma^k\end{bmatrix}
& \mu = i, \nu = j.
\end{cases}
\end{align}

%%%%%%%%%%%%%%%%%%%%%%%%%%%%%%%%
\section{Fresnel functional integral}

%%%%%%%%%%%%%%%%%%%%%%%%%%%%%%%%

\citep[ch.\ 10]{mosel2004}

\section{Algebra}
\begin{align}
\sbr{a\partial_1 \partial_2, x^1}_- = a\partial_2
\end{align}
central.

Baker--Campbell--Hausdorff formula

\begin{align}
\me^{+a\partial_1\partial_2} x^1 \me^{-a\partial_1\partial_2}
= x^1 + a \partial_2.
\end{align}

%%%%%%%%%%%%%%%%%%%%%%%%%%%%%%%%
\section{Schwinger integral}

%%%%%%%%%%%%%%%%%%%%%%%%%%%%%%%%

\begin{align}
-\int_{\epsilon}^{+\infty} \frac{\dif t}{t}\,\me^{-\alpha t} =
-\rfun{\Gamma}{0, \alpha \epsilon} = \upgamma_\text{E} + \ln\alpha + 
\ln\epsilon + \rfun{O}{\epsilon},
\end{align}
where $\rfun{\Gamma}{a,z}$ is the incomplete Gamma function, 
$\upgamma_\text{E}$ is Euler's constant.


\bibliographystyle{plainnat}
\bibliography{./eft}
%\printbibliography

\end{document}
