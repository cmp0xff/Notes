\documentclass[a4paper,11pt]{article}
%\documentclass[a4\mfrakpa\mfrakper,11\mfrakpt]{scrartcl}



% Fonts and languages

% Multilingual support
%\usepackage{polyglossia}

% more symbols
\usepackage{textcomp}

% AMS--related
\usepackage{amsmath,amssymb}

% ':=' as \coloneqq
\usepackage{mathtools}
% Physical bras and kets
\usepackage{braket}
% SI units
\usepackage{siunitx}
\sisetup{separate-uncertainty}

\usepackage{graphicx}
%\usepackage[colorinlistoftodos]{todonotes}


% Boxed equations. NEED TO BE LOADED BEFORE unicode-math!
\usepackage{empheq}
% Theorems
\usepackage{amsthm}

% Chemical elements
%\usepackage[version=4]{mhchem}


%\usepackage{fancybox}

\usepackage{enumitem} % \begin{enumerate}[label=\Alph*]


% Other formats

% Labelling equations according to sections
\numberwithin{equation}{section}


% Bibliography in the main text!!!

%\usepackage[
			%style=alphabetic,
%			backend=biber]{biblatex}
%\usepackage{hyperref}
\usepackage{nameref}

% Cross-references
% The \newtheorem commands have to come after the loading of {cleveref}.
% Additionally, the cleverref package has to be loaded after ntheorem or
% amsthm. cleverref has to be loaded after hyperref!
\usepackage{cleveref}
%\usepackage{nameref}%,thmtools}

% Plot
\usepackage{tikz}
\usetikzlibrary{decorations.pathmorphing}
\usetikzlibrary{calc}
% save compiled tikz plots; enable --shell-escape
%\usetikzlibrary{external}
%\tikzexternalize[prefix=./tikz/]

% Fontspec fot XeLaTeX
\usepackage{fontspec}
	% Unicode fonts
	\setmainfont{CMU Serif}
	\setsansfont{CMU Sans Serif}
	\setmonofont{CMU Typewriter Text}
	% declare a command \doulos to load the Doulos SIL font
	%\newfontfamily\brill{Brill}
	% now create a \textIPA{} command
	%\DeclareTextFontCommand{\textIPA}{\brill}
\usepackage{amsfonts}
\usepackage{unicode-math}
\usepackage{unicode-math}
	\setmathfont{Latin Modern Math} % default
	%\setmathfont[range=\mathalpha]{Asana Math}
	\setmathfont{Asana Math}[range={\mathbin}] %\mathord
	\setmathfont{STIX Math}[range={"02609}] % ☉
	\setmathfont{XITS Math}[range={"1D4B6-"1D4CF}] % Script, Latin, lowercase
	\setmathfont{Latin Modern Math}[range={"1D608-"1D63B}, sans-style=italic]
	\setmathfont{Latin Modern Math}[range={
		"00391-"003A9,
		"003B1-"003F5, 
		"1D6A8-"1D6E1},	% Bold Greek
		sans-style=upright]
	%\setmathfont{⟨font name⟩}[range=⟨unicode range⟩,⟨font features⟩]

\input{../preambles/unicode}

\setmainlanguage{english}
\setotherlanguages{german,greek,russian}

\input{../preambles/math-single}
\input{../preambles/math-brac}
\input{../preambles/math-thm}
\input{../preambles/phys-chem}

% \setromanfont[Mappping=tex-text]{Linux Libertine O}
% \setsansfont[Mapping=tex-text]{DejaVu Sans}
% \setmonofont[Mapping=tex-text]{DejaVu Sans Mono}

\usepackage[%style=authoryear-icomp,
			backend=biber]{biblatex}
\addbibresource{./cosmo-perturb.bib}

\title{Cosmological Perturbations}
\author{Yi-Fan Wang (王\ 一帆)}
%\date{}

\begin{document}
\maketitle

Most of the conventions and notations in \cite[ch.~5]{Weinberg2008} will be 
followed.

Suppose the metric can be expanded up to the linear 
order as
\begin{align}
g = g^{(0)} + \epsilon g^{(1)} + \rfun{\Omicron}{\epsilon^2}.
\end{align}
The background metric $g^{(0)}$ takes the Robertson--Walker form
\begin{align}
g^{(0)}_{\mu\nu}\,\dd x^\mu\,\dd x^\nu =
-\rfun{N^2}{t}\,\dd t^2 + \rfun{a^2}{t}\,\dd \Omega_\text{3F}^2,
\end{align}
in which $\dd\Omega_\text{3F}^2 =
\dd\chi^2 + \chi^2\rbr{\dd\theta^2 + \sin^2\theta\,\dd \phi^2}$ is the 
dimensionless flat spatial metric.
The linear perturbation can be decomposed into scalar, vector and 
tensor parts
\begin{align}
g^{(1)}_{00} &= -E, \\
g^{(1)}_{i0} = g^{(1)}_{i0} &= F_{,i}+G_i, \\
g^{(1)}_{ij} &= A \delta_{ij} + B_{,i,j}+C_{i,j}+C_{j,i}+D_{ij}.
\end{align}

Here one has some weird condition, where the contractions do not follow the 
one-up-one-down tradition
\begin{align}
C_{i,i} = G_{i,i} = 0,
\quad
D_{ij,i} = 0,
\quad
D_{ii} = 0.
\end{align}

%%%%%%%%%%%%%%%%%%%%%%%%%%%%%%%%
\section{Metric perturbation under diffeomorphism}

%%%%%%%%%%%%%%%%%%%%%%%%%%%%%%%%

Consider a diffeomorphism generated by $\xi^\mu$
\begin{align}
x^\mu \to \overline{x}^\mu = x^\mu - \epsilon \xi^\mu.
\end{align}
The generator $\xi^\mu$ can in turn be decomposed into $\xi_0 = \zeta$, 
$\xi_i 
= \xi^\text{S}_{,i} + \xi^\text{V}_{i}$.

One has here again some weird condition, where the contractions do not follow 
the one-up-one-down tradition
\begin{align}
\xi^\text{V}_{i,i} = 0.
\end{align}

The Lie derivative of the metric $\BbbL_{\xi} g$ is
\begin{align}
\rbr{\BbbL_{\xi} g}_{\mu\nu} =
\xi^\lambda g_{\mu\nu,\lambda} + 
\xi^{\lambda}{}_{,\mu} g_{\lambda\nu} +
\xi^{\lambda}{}_{,\nu} g_{\mu\lambda}.
\end{align}
In components and expansion, these are
\begin{align}
\rbr{\BbbL_{\xi} g}_{00} &=
2\dot\zeta - 2\zeta \frac{\dot{N}}{N} + \rfun{\Omicron}{\epsilon}, \\
\rbr{\BbbL_{\xi} g}_{i0} = \rbr{\BbbL_{\xi} g}_{0i} &=
\rbr{\zeta-2\frac{\dot{a}}{a}\xi^\text{S}+\dot{\xi}^\text{S}}_{,i} +
\rbr{-2\frac{\dot{a}}{a}\xi^\text{V}_{i}+\dot{\xi}^\text{V}_{i}}
+ \rfun{\Omicron}{\epsilon}, \\
\rbr{\BbbL_{\xi} g}_{ji} = \rbr{\BbbL_{\xi} g}_{ij} &=
-\frac{2a\dot{a}}{N^2} \zeta \delta_{ij} + 2\xi^\text{S}_{,i,j}
+ \xi^\text{V}_{i,j} + \xi^\text{V}_{j,i} + \rfun{\Omicron}{\epsilon}.
\end{align}

%%%%%%%%%%%%%%%%%%%%%%%%%%%%%%%%
\section{Scalar perturbations}

%%%%%%%%%%%%%%%%%%%%%%%%%%%%%%%%


\begin{align}
-N^2 - \epsilon E + \rfun{\Omicron}{\epsilon^2} \to
-N^2 - \epsilon E + \epsilon\rbr{2\dot\zeta - 
2\zeta\frac{\dot{N}}{N}}
+ \rfun{\Omicron}{\epsilon^2},
\end{align}
so one can write
\begin{align}
\BbbL_\xi E = -2 \dot{\zeta} + 2 \zeta \frac{\dot{N}}{N}.
\end{align}
Similarly one can read-off
\begin{align}
\BbbL_\xi F &= \zeta - 2\frac{\dot{a}}{a}\xi^\text{S} + \dot{\xi}^\text{S}, \\
\BbbL_\xi A &= -\frac{2a\dot{a}}{N^2}\zeta, \\
\BbbL_\xi B &= 2\xi^\text{S}.
\end{align}

The four scalar perturbations are generated by $\zeta$ and $\xi^\text{S}$, 
so that only two independent perturbations exists. It is clear that
\begin{align}
\BbbL_\xi\rbr{\frac{F}{a}-\frde{}{t}\frac{B}{2a}} = \frac{\zeta}{a}.
\end{align}
One can verify that
\begin{align}
\BbbL_\xi\cbr{\frac{E}{2N}+
\frde{}{t}\sbr{\frac{a}{N}\rbr{\frac{F}{a}-\frde{}{t}\frac{B}{2a}}}} &= 0, \\
\BbbL_\xi\cbr{\frac{A}{2}+
\frac{a^2 \dot{a}}{N^2}\rbr{\frac{F}{a}-\frde{}{t}\frac{B}{2a}}} &= 0.
\end{align}




%%%%%%%%%%%%%%%%%%%%%%%%%%%%%%%%
\section{Vector perturbations}

%%%%%%%%%%%%%%%%%%%%%%%%%%%%%%%%


%%%%%%%%%%%%%%%%%%%%%%%%%%%%%%%%
\section{Tensor perturbations}

%%%%%%%%%%%%%%%%%%%%%%%%%%%%%%%%


%%%%%%%%%%%%%%%%%%%%%%%%%%%%%%%%
\section{Scalar field perturbation under diffeomorphism}

%%%%%%%%%%%%%%%%%%%%%%%%%%%%%%%%

%%%%%%%%%%%%%%%%%%%%%%%%%%%%%%%%
\section{Perturbation in Arnowitt--Deser--Misner Hamiltonian formalism}

%%%%%%%%%%%%%%%%%%%%%%%%%%%%%%%%

Up to boundary terms, the Hamiltonian action for General Relativity in terms 
of Arnowitt--Deser--Misner variables is \cite[ch.4.2.2]{Kiefer2012}
\begin{align}
S &=
\int\dd t\,\dd x^3\,\cbr{
\mfrakp^{ij}\dot{h}_{ij} + \mfrakP\dot{N} + \mfrakP^i \dot{N}_i
-N\mfrakH^\perp - N_i\mfrakH^i - \mfrakP V - \mfrakP^i V_i}, \\
\mfrakH^\perp &=
2 \varkappa \mfrakG_{ijkl} \mfrakp^{ij}\mfrakp^{kl}
- \frac{\sqrt{\mfrakh}}{2\varkappa} \sfun{R}{h}
=
2 \varkappa \mfrakF^{ijkl} h_{ij}h_{kl}
- \frac{\sqrt{\mfrakh}}{2\varkappa} \sfun{R}{h},
\\
\mfrakH^i &=
-2 \mfrakp^{ij}{}_{|j},
\\
\mfrakG_{ijkl} &\coloneqq \frac{1}{2\sqrt{\mfrakh}}
\rbr{h_{ik}h_{jl}+h_{il}h_{kj}-h_{ij}h_{kl}},
\\
\mfrakF^{ijkl} &\coloneqq \frac{1}{2\sqrt{\mfrakh}}
\rbr{\mfrakp^{ik}\mfrakp^{jl}+\mfrakp^{il}\mfrakp^{kj}
-\mfrakp^{ij}\mfrakp^{kl}},
\end{align}
where $V$ and $V_i$ are velocities of $N$ and $N_i$ and play the role of 
Lagrange multipliers. Technical details about the boundary terms can be 
found in \cite[ch.\ 4.2]{Poisson2004} and the references therein. Note that 
$\cbr{N, N_i, h_{ij}; \mfrakP, \mfrakP^i, \mfrakp^{ij}}$ are not the unique 
choice of canonical variables for General Relativity in Hamiltonian formalism; 
instead, they are a special parametrisation of the phase space. One can also 
choose the components of the original four-metric and their conjugate momenta 
$\cbr{g_{\mu\nu}; \mfrakp^{\mu\nu}}$ as canonical variables, as 
\citeauthor{Dirac1958} has done in \cite{Dirac1958}. The two approaches are 
different in some subtle aspects; see 
\cite{Kiriushcheva2008} for a comparison.

Gauge transformations in the Arnowitt--Deser--Misner canonical variables are 
generated by 
\cite{Castellani1982} 
\begin{align}
G &= -\int\dd^3 x\,\Big\{ \sbr{
	\xi_\perp \rbr{
		\mfrakH^\perp + N_{|i} \mfrakP^i + \rbr{N\mfrakP^i}_{|i} + 
			\rbr{N_i \mfrakP}^{|i}} +
	\dot{\xi}_\perp \mfrakP}
	\nonumber \\ &\qquad\qquad\ \,+ \sbr{
	\xi_i \rbr{
		\mfrakH^i + N_j{}^{|i} \mfrakP^j + \rbr{N_j \mfrakP^i}^{|j} + N^{|i} 
\mfrakP} +
	\dot{\xi}_i \mfrakP^i} \Big\}.
\end{align}
Possible boundary terms have not been discussed so far. The infinitesimal gauge 
transformation of $\cbr{N, N_i}$ is
\begin{align}
\dva N &= \sbr{N, G}_\text{P} = 
\xi_\perp{}^{|i} N_zi - \dot{\xi}_\perp - \xi_i N^{|i}, \\
\dva N_i &= %\sbr{N_i, G}_\text{P} =
- \xi_\perp N_{|i} + \xi_\perp{}_{|i} N
- \xi_j N_{i}{}^{|j} + \xi_i{}^{|j} N_j - \dot{\xi}_i,
\end{align}
which can be found in \cite{Kiriushcheva2008}. Transformations for 
$g_{ij}$ and the momenta have to be worked out as
\begin{align}
\dva \mfrakP &= -\rbr{\xi_\perp \mfrakP^i}_{|i} - \xi_\perp{}_{|i} \mfrakP^i
- \rbr{\xi_i \mfrakP}^{|i},\\
\dva \mfrakP^i &= -\xi_\perp{}_{|i} \mfrakP
- \rbr{\xi_j \mfrakP^i}^{|j} - \xi_j{}^{|i}\mfrakP^j,
\end{align}
where only the primary constraints are involved;
\begin{align}
\dva g_{ij} &= -\frpa{}{\mfrakp^{ij}}
\rbr{\xi_\perp \mfrakH^\perp + \xi_i \mfrakH^i}
\nonumber \\
&= -\xi^\perp \frac{2\varkappa}{\sqrt{\mfrakh}}
\rbr{h_{ik}h_{jl}+h_{il}h_{kj}-h_{ij}h_{kl}} \mfrakp^{kl}
- \xi_{i|j} - \xi_{j|i}, \\
\dva \mfrakp^{ij} &= \frpa{}{g_{ij}}
\rbr{\xi_\perp \mfrakH^\perp + \xi_i \mfrakH^i}
\nonumber \\
&= \xi_\perp \bigg\{\frac{\varkappa}{\sqrt{\mfrakh}}\bigg[
\rbr{4\mupdelta^i{}_k \mupdelta^j{}_m - h^{ij} h_{km}}h_{ln}
\nonumber \\
&\qquad\quad\ \,
-\frac{1}{2}
\rbr{4\mupdelta^i{}_k \mupdelta^j{}_l - h^{ij} h_{kl}}h_{mn}
\bigg] \mfrakp^{kl}\mfrakp^{mn}
\nonumber \\
&\qquad\,
+ \frac{\sqrt{\mfrakh}}{2\varkappa} \sfun{G^{ij}}{h} \bigg\}
+ \frac{\sqrt{\mfrakh}}{2\varkappa} \rbr{
\xi_\perp{}_{|k}{}^{|k} h^{ij} - \xi_\perp^{|(i|j)}}
\nonumber \\
&\quad\,
-\rbr{\xi_k{} \mfrakp^{ij}}^{|k} + 2\xi_{k|l}h^{k(i} \mfrakp^{j)l},
\label{eq:var-mom}
\end{align}
where $\sfun{G^{ij}}{h} = \sfun{R^{ij}}{h} - h^{ij}\sfun{R}{h}/2$, and only the 
secondary constraints are involved. In \cref{eq:var-mom}, the first two lines
come from the variation of the `kinetic' term in $\mfrakH^\perp$, the third
comes from the `potential' term in $\mfrakH^\perp$, and the last line from
$\mfrakH^i$. The results can be checked with \cite[4.2.7]{Poisson2004}.

\subsection{Expansion of the action with fluctuations}

\appendix

\section*{Some useful results}

\subsection*{First variations}

First variation of $h^{ij}$
\begin{align}
\dva h^{ij} = -h^{ik}h^{jl}\,\dva h_{kl} = - h^{i(k}h^{l)j}\,\dva h_{kl}.
\end{align}

First variation of $\mfrakh = \det h_{ij}$
\begin{align}
\dva \mfrakh = \mfrakh h^{ij}\,\dva h_{ij}.
\end{align}

First variation of $\Gamma^i{}_{jk}$
\begin{align}
\dva \Gamma^{i}{}_{jk} &= \frac{1}{2} h^{il}
\cbr{-\rbr{\dva h_{jk}}_{|l} + \rbr{\dva h_{kl}}_{|j} + \rbr{\dva h_{lj}}_{|k}}
\\
&= \frac{1}{2} \cbr{-h^{il}\mupdelta^m{}_j\mupdelta^n{}_k
+h^{in}\mupdelta^l{}_j\mupdelta^m{}_k
+h^{im}\mupdelta^n{}_j\mupdelta^l{}_k}\,\rbr{\dva h_{mn}}_{|l}.
\end{align}

First variation of $\sfun{R_{ij}}{h}$ and $\sfun{R^{ij}}{h}$
\begin{align}
\dva \sfun{R_{ij}}{h} &=
\rbr{\dva\Gamma^k{}_{ji}}_{|k} - \rbr{\dva\Gamma^k{}_{ki}}_{|j}
\label{eq:var-R_{ij}}
\\
&=
\frac{1}{2}\rbr{\dva h_{kl}}_{|m|n}\,\big(
\mupdelta^k{}_ih^{ln}\mupdelta^m{}_j+
\mupdelta^m{}_i\mupdelta^l{}_jh^{kn}
\nonumber \\
&\qquad\qquad
-\mupdelta^k{}_i\mupdelta^l{}_jh^{mn}
-\mupdelta^m{}_i\mupdelta^n{}_jh^{kl}\big),
\\
\dva \sfun{R^{ij}}{h} &=
-2R^{ki}h^{jl}\,\dva h_{kl} + h^{ik}\,\bar{\dva}u^{jl}{}_{k|l},
\end{align}
where
\begin{align}
\bar{\dva}u^{ij}{}_{k} &\coloneqq
h^{il}\,\dva\Gamma^{j}{}_{kl} - h^{ij}\,\dva\Gamma^{l}{}_{kl}
\\
&= 
\frac{1}{2}\rbr{\dva h_{mn}}_{|l}\,\big\{
-\mupdelta^m{}_k\rbr{h^{in}h^{jl}-h^{ij}h^{ln}}
\nonumber \\
&\qquad\qquad
+\mupdelta^l{}_k\rbr{h^{im}h^{jn}-h^{ij}h^{mn}}
+\mupdelta^n{}_k\rbr{h^{il}h^{jm}-h^{ij}h^{lm}}\big\}.
\end{align}
\Cref{eq:var-R_{ij}} can be obtained by using normal coordinates.

First variation of $\sqrt{\mfrakh}\,\sfun{R}{h}$
\begin{align}
\dva\rbr{\sqrt{\mfrakh}\sfun{R}{h}} = \sqrt{\mfrakh}\,
\cbr{-\sfun{G^{ij}}{h}\,\dva h_{ij} +
\bar{\dva}u^{ji}{}_{j|i}}.
\end{align}
The $\bar\dva$ term contains second derivatives of $\dva h_{ij}$, which is not 
desired. Being a covariant divergence of a vector, it can be pushed to the 
boundary upon variation of the action, and will be cancelled by the boundary 
terms.

First variation of $\mfrakG_{ijkl}\mfrakp^{ij}\mfrakp^{kl} \equiv
\mfrakF^{ijkl}h_{ij}h_{kl}$
\begin{align}
&\quad\,\dva\rbr{ \mfrakG_{ijkl} \mfrakp^{ij}\mfrakp^{kl}}
\equiv \dva\rbr{\mfrakF^{ijkl}h_{ij}h_{kl}}
\nonumber \\
&=\dva h_{ij}\,\rbr{-\frac{1}{2} h^{ij} \mfrakF^{klmn} h_{mn}
+ 2 \mfrakF^{ijkl} } h_{kl} 
+\dva \mfrakp^{ij}\,2 \mfrakG_{ijkl}\mfrakp^{kl}.
\end{align}

First variation of $\mfrakH^\perp$
\begin{align}
\dva \mfrakH^\perp &= \dva h_{ij}\,\bigg\{
2\varkappa \rbr{ -\frac{1}{2}h^{ij} \mfrakF^{klmn} h_{mn}
+ 2 \mfrakF^{ijkl} } h_{kl}
+\frac{\sqrt{\mfrakh}}{2\varkappa}\sfun{G^{ij}}{h}\bigg\}
\nonumber \\
&\qquad
+\dva \mfrakp^{ij}\,4\varkappa \mfrakG_{ijkl}\mfrakp^{kl}
-\frac{\sqrt{\mfrakh}}{2\varkappa} \bar{\dva}u^{ji}{}_{j|i}} 
\end{align}
where the terms with a $\sqrt{\mfrakh}/2\varkappa$ factor come from variation 
of the three Ricci scalar.
% and the $\bar{\dva}$ term is expected to be cancelled by the boundary terms.

First variation of $\mfrakH^i$
\begin{align}
\dva \mfrakH^i = \rbr{\dva p^{ij}}_{|j}
+ \dva\Gamma^{i}{}_{jk}\,p^{jk}.
\end{align}

First variation of $\mfrakF^{ijkl}h_{kl}$
\begin{align}
\dva\rbr{\mfrakF^{ijkl}h_{kl}} &= \dva h_{kl}\,
\rbr{-\frac{1}{2}\mfrakF^{ijmn}h^{kl}h_{mn}+\mfrakF^{ijkl}}
\nonumber \\
&\quad\,
+\dva \mfrakp^{kl}\,\rbr{
\mupdelta^{i}{}_{k}\mfrakG^{j}{}_{lmn} +
\mupdelta^{i}{}_{m}\mfrakG^{j}{}_{nkl}}\mfrakp^{mn}.
\end{align}

First variation of $\mfrakG_{ijkl}\mfrakp^{kl}$
\begin{align}
\dva\rbr{\mfrakG_{ijkl}\mfrakp^{kl}} &= \dva h_{kl}\,
\rbr{-\frac{1}{2}\mfrakG_{ijmn}h^{kl}+
\mupdelta^{k}{}_{i}\mfrakG^{l}{}_{jmn}+
\mupdelta^{k}{}_{m}\mfrakG^{l}{}_{nij}}\mfrakp^{mn}
\nonumber \\
&\quad\,
+\dva p^{kl}\,\mfrakG_{ijkl}.
\end{align}

First variation of $\sqrt{\mfrakh}\,\sfun{G^{ij}}{h}$
\begin{align}
&
\dva\rbr{\sqrt{\mfrakh}\,\sfun{G^{ij}}{h}}
= \sqrt{\mfrakh}\bigg\{\dva h_{kl}\cdot
\nonumber \\
&\qquad\quad
\rbr{-\frac{1}{2}}\rbr{
R^{ik}h^{lj}+R^{il}h^{kj}-R^{ij}h^{kl}+h^{ik}G^{lj}+h^{il}G^{kj}-h^{ij}G^{kl}}
\nonumber \\
&\quad
+\rbr{h^{il}\,\bar{\dva}u^{jk}{}_{l}
- \frac{1}{2}h^{ij}\,\bar{\dva}u^{lk}{}_{l}}_{|k} \bigg\}.
\end{align}

\subsection*{Second variations}

Second variation of $\mfrakH^\perp$
\begin{align}
&\quad\,
\dva^2\rbr{\mfrakH^\perp}
\nonumber \\
&= \dva h_{ij}\,\dva h_{kl}\,\bigg\{ 2\varkappa\bigg[
\frac{1}{4}\rbr{h^{ik}h^{lj}+h^{il}h^{kj}+h^{ij}h^{kl}} \mfrakF^{mnrs} 
h_{mn} h_{rs}
\nonumber \\
&\qquad\qquad\qquad
-\rbr{h^{ij}\mfrakF^{klmn}+\mfrakF^{ijmn}h^{kl}}h_{mn}
+\mfrakF^{ijkl} \bigg]
\nonumber \\
&\qquad\quad
-\frac{\sqrt{\mfrakh}}{4\varkappa}\rbr{
R^{ik}h^{lj}+R^{il}h^{kj}-R^{ij}h^{kl}+h^{ik}G^{lj}+h^{il}G^{kj}-h^{ij}G^{kl}}
\bigg\}
\nonumber \\
&\quad\,
+\dva h_{ij}\,\frac{\sqrt{\mfrakh}}{2\varkappa}
\rbr{h^{il}\,\bar{\dva}u^{jk}{}_{l}
- \frac{1}{2}h^{ij}\,\bar{\dva}u^{lk}{}_{l}}_{|k} 
\nonumber \\
&\quad\,
+\dva h_{ij}\,\dva \mfrakp^{kl}\,4\varkappa
\cbr{-h^{ij}\mfrakG_{klmn}+2\rbr{\mupdelta^i{}_k\mfrakG^{j}{}_{lmn} 
+\mupdelta^i{}_m\mfrakG^{j}{}_{nkl}} } \mfrakp^{mn}
\nonumber \\
&\quad\,
+\dva p^{ij}\,\dva p^{kl}\,4\varkappa \mfrakG_{ijkl}
\nonumber \\
&\quad\,
-\dva h_{ij}\,
\dva\rbr{\,\frac{\sqrt{\mfrakh}}{2\varkappa}\,\bar{\dva}u^{lk}{}_{l|k}}.
\end{align}

Second variation of $\Gamma^i{}_{jk}$
\begin{align}
\dva^2 \Gamma^i{}_{jk} &=
- h^{im} \,\dva\Gamma^{l}{}_{jk} \,\dva h_{lm}.
\end{align}

Second variation of $\mfrakH^i$
\begin{align}
\dva^2\rbr{\mfrakH^i} = -h^{im}p^{jk}
\,\dva\Gamma^{l}{}_{jk}\,\dva h_{lm} + 2\,\dva\Gamma^{i}{}_{jk}\,\dva p^{jk}.
\end{align}


\subsection*{Other second variations}

Second variation of $h^{ij}$
\begin{align}
\dva^2 h^{ij} = \rbr{h^{im}h^{jl}h^{kn}+h^{ik}h^{jm}h^{ln}}
\,\dva h_{kl}\,\dva h_{mn}
\end{align}

Second variation of $\mfrakh = \det h_{ij}$
\begin{align}
\dva^2 \mfrakh = -\frac{1}{4} \mfrakh \rbr{
h^{ik}h^{jl}+h^{il}h^{kj}-h^{ij}h^{kl}}\,\dva h_{ij}\,\dva h_{kl}.
\end{align}

First variation of $\rbr{\dva h_{ij}}_{|k}$
\begin{align}
\dva\cbr{\rbr{\dva h_{ij}}_{|k}} = -2 \dva\Gamma^l{}_{k(i}\,\dva h_{j)l}.
\end{align}

% Second variation of $\sqrt{\mfrakh}\sfun{R}{h}$
% \begin{align}
% &
% \dva^2\rbr{\sqrt{\mfrakh}R} = \sqrt{\mfrakh} \bigg\{
% \dva h_{ij}\,\dva h_{kl}\bigg[
% \frac{1}{4}R\rbr{h^{ik}h^{jl}+h^{il}h^{jk}-h^{ij}h^{kl}}
% \nonumber \\
% &\qquad
% +\frac{1}{2}\rbr{
% R^{ij}h^{kl}+R^{kl}h^{ij}-R^{ik}h^{jl}-R^{il}h^{jk}-R^{jk}h^{il}-R^{il}h^{kj}}
% \bigg]
% \nonumber \\
% &\quad
% +h^{ik}\,\bar{\dva}u^{jl}{}_{k|l}+\dva\rbr{\bar{\dva}u^{ji}{}_{j|i}} \bigg\}
% \end{align}


In a general background, the second variations of the quantities are much more 
tedious.

In Robertson--Walker background, 


\printbibliography

\end{document}
