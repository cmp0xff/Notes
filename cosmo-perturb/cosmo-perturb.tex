\documentclass[a4paper,11pt]{article}
%\documentclass[a4\mfrakpa\mfrakper,11\mfrakpt]{scrartcl}



% Fonts and languages

% Multilingual support
%\usepackage{polyglossia}

% more symbols
\usepackage{textcomp}

% AMS--related
\usepackage{amsmath,amssymb}

% ':=' as \coloneqq
\usepackage{mathtools}
% Physical bras and kets
\usepackage{braket}
% SI units
\usepackage{siunitx}
\sisetup{separate-uncertainty}

\usepackage{graphicx}
%\usepackage[colorinlistoftodos]{todonotes}


% Boxed equations. NEED TO BE LOADED BEFORE unicode-math!
\usepackage{empheq}
% Theorems
\usepackage{amsthm}

% Chemical elements
%\usepackage[version=4]{mhchem}


%\usepackage{fancybox}

\usepackage{enumitem} % \begin{enumerate}[label=\Alph*]


% Other formats

% Labelling equations according to sections
\numberwithin{equation}{section}


% Bibliography in the main text!!!

%\usepackage[
			%style=alphabetic,
%			backend=biber]{biblatex}
%\usepackage{hyperref}
\usepackage{nameref}

% Cross-references
% The \newtheorem commands have to come after the loading of {cleveref}.
% Additionally, the cleverref package has to be loaded after ntheorem or
% amsthm. cleverref has to be loaded after hyperref!
\usepackage{cleveref}
%\usepackage{nameref}%,thmtools}

% Plot
\usepackage{tikz}
\usetikzlibrary{decorations.pathmorphing}
\usetikzlibrary{calc}
% save compiled tikz plots; enable --shell-escape
%\usetikzlibrary{external}
%\tikzexternalize[prefix=./tikz/]

% Fontspec fot XeLaTeX
\usepackage{fontspec}
	% Unicode fonts
	\setmainfont{CMU Serif}
	\setsansfont{CMU Sans Serif}
	\setmonofont{CMU Typewriter Text}
	% declare a command \doulos to load the Doulos SIL font
	%\newfontfamily\brill{Brill}
	% now create a \textIPA{} command
	%\DeclareTextFontCommand{\textIPA}{\brill}
\usepackage{amsfonts}
\usepackage{unicode-math}
\usepackage{unicode-math}
	\setmathfont{Latin Modern Math} % default
	%\setmathfont[range=\mathalpha]{Asana Math}
	\setmathfont{Asana Math}[range={\mathbin}] %\mathord
	\setmathfont{STIX Math}[range={"02609}] % ☉
	\setmathfont{XITS Math}[range={"1D4B6-"1D4CF}] % Script, Latin, lowercase
	\setmathfont{Latin Modern Math}[range={"1D608-"1D63B}, sans-style=italic]
	\setmathfont{Latin Modern Math}[range={
		"00391-"003A9,
		"003B1-"003F5, 
		"1D6A8-"1D6E1},	% Bold Greek
		sans-style=upright]
	%\setmathfont{⟨font name⟩}[range=⟨unicode range⟩,⟨font features⟩]

\input{../preambles/unicode}

\setmainlanguage{english}
\setotherlanguages{german,greek,russian}

\input{../preambles/math-single}
\input{../preambles/math-brac}
\input{../preambles/math-thm}
\input{../preambles/phys-chem}

% \setromanfont[Mappping=tex-text]{Linux Libertine O}
% \setsansfont[Mapping=tex-text]{DejaVu Sans}
% \setmonofont[Mapping=tex-text]{DejaVu Sans Mono}

\usepackage[%style=authoryear-icomp,
			backend=biber]{biblatex}
\addbibresource{./cosmo-perturb.bib}

\title{Cosmological Perturbations}
\author{Yi-Fan Wang (王\ 一帆)}
%\date{}

\begin{document}
\maketitle

Most of the conventions and notations in \cite[ch.~5]{Weinberg2008} will be 
followed.

Suppose the metric can be expanded up to the linear 
order as
\begin{align}
g = g^{(0)} + \epsilon g^{(1)} + \rfun{\Omicron}{\epsilon^2}.
\end{align}
The background metric $g^{(0)}$ takes the Robertson--Walker form
\begin{align}
g^{(0)}_{\mu\nu}\,\dd x^\mu\,\dd x^\nu =
-\rfun{N^2}{t}\,\dd t^2 + \rfun{a^2}{t}\,\dd \Omega_\text{3F}^2,
\end{align}
in which $\dd\Omega_\text{3F}^2 =
\dd\chi^2 + \chi^2\rbr{\dd\theta^2 + \sin^2\theta\,\dd \phi^2}$ is the 
dimensionless flat spatial metric.
The linear perturbation can be decomposed into scalar, vector and 
tensor parts
\begin{align}
g^{(1)}_{00} &= -E, \\
g^{(1)}_{i0} = g^{(1)}_{i0} &= F_{,i}+G_i, \\
g^{(1)}_{ij} &= A \delta_{ij} + B_{,i,j}+C_{i,j}+C_{j,i}+D_{ij}.
\end{align}

Here one has some weird condition, where the contractions do not follow the 
one-up-one-down tradition
\begin{align}
C_{i,i} = G_{i,i} = 0,
\quad
D_{ij,i} = 0,
\quad
D_{ii} = 0.
\end{align}

%%%%%%%%%%%%%%%%%%%%%%%%%%%%%%%%
\section{Metric perturbation under diffeomorphism}

%%%%%%%%%%%%%%%%%%%%%%%%%%%%%%%%

Consider a diffeomorphism generated by $\xi^\mu$
\begin{align}
x^\mu \to \overline{x}^\mu = x^\mu - \epsilon \xi^\mu.
\end{align}
The generator $\xi^\mu$ can in turn be decomposed into $\xi_0 = \zeta$, 
$\xi_i 
= \xi^\text{S}_{,i} + \xi^\text{V}_{i}$.

One has here again some weird condition, where the contractions do not follow 
the one-up-one-down tradition
\begin{align}
\xi^\text{V}_{i,i} = 0.
\end{align}

The Lie derivative of the metric $\BbbL_{\xi} g$ is
\begin{align}
\rbr{\BbbL_{\xi} g}_{\mu\nu} =
\xi^\lambda g_{\mu\nu,\lambda} + 
\xi^{\lambda}{}_{,\mu} g_{\lambda\nu} +
\xi^{\lambda}{}_{,\nu} g_{\mu\lambda}.
\end{align}
In components and expansion, these are
\begin{align}
\rbr{\BbbL_{\xi} g}_{00} &=
2\dot\zeta - 2\zeta \frac{\dot{N}}{N} + \rfun{\Omicron}{\epsilon}, \\
\rbr{\BbbL_{\xi} g}_{i0} = \rbr{\BbbL_{\xi} g}_{0i} &=
\rbr{\zeta-2\frac{\dot{a}}{a}\xi^\text{S}+\dot{\xi}^\text{S}}_{,i} +
\rbr{-2\frac{\dot{a}}{a}\xi^\text{V}_{i}+\dot{\xi}^\text{V}_{i}}
+ \rfun{\Omicron}{\epsilon}, \\
\rbr{\BbbL_{\xi} g}_{ji} = \rbr{\BbbL_{\xi} g}_{ij} &=
-\frac{2a\dot{a}}{N^2} \zeta \delta_{ij} + 2\xi^\text{S}_{,i,j}
+ \xi^\text{V}_{i,j} + \xi^\text{V}_{j,i} + \rfun{\Omicron}{\epsilon}.
\end{align}

%%%%%%%%%%%%%%%%%%%%%%%%%%%%%%%%
\section{Scalar perturbations}

%%%%%%%%%%%%%%%%%%%%%%%%%%%%%%%%


\begin{align}
-N^2 - \epsilon E + \rfun{\Omicron}{\epsilon^2} \to
-N^2 - \epsilon E + \epsilon\rbr{2\dot\zeta - 
2\zeta\frac{\dot{N}}{N}}
+ \rfun{\Omicron}{\epsilon^2},
\end{align}
so one can write
\begin{align}
\BbbL_\xi E = -2 \dot{\zeta} + 2 \zeta \frac{\dot{N}}{N}.
\end{align}
Similarly one can read-off
\begin{align}
\BbbL_\xi F &= \zeta - 2\frac{\dot{a}}{a}\xi^\text{S} + \dot{\xi}^\text{S}, \\
\BbbL_\xi A &= -\frac{2a\dot{a}}{N^2}\zeta, \\
\BbbL_\xi B &= 2\xi^\text{S}.
\end{align}

The four scalar perturbations are generated by $\zeta$ and $\xi^\text{S}$, 
so 
that only two independent perturbations exists. It is clear that
\begin{align}
\BbbL_\xi\rbr{\frac{F}{a}-\frde{}{t}\frac{B}{2a}} = \frac{\zeta}{a}.
\end{align}
One can verify that
\begin{align}
\BbbL_\xi\cbr{\frac{E}{2N}+
\frde{}{t}\sbr{\frac{a}{N}\rbr{\frac{F}{a}-\frde{}{t}\frac{B}{2a}}}} &= 0, \\
\BbbL_\xi\cbr{\frac{A}{2}+
\frac{a^2 \dot{a}}{N^2}\rbr{\frac{F}{a}-\frde{}{t}\frac{B}{2a}}} &= 0.
\end{align}




%%%%%%%%%%%%%%%%%%%%%%%%%%%%%%%%
\section{Vector perturbations}

%%%%%%%%%%%%%%%%%%%%%%%%%%%%%%%%


%%%%%%%%%%%%%%%%%%%%%%%%%%%%%%%%
\section{Tensor perturbations}

%%%%%%%%%%%%%%%%%%%%%%%%%%%%%%%%


%%%%%%%%%%%%%%%%%%%%%%%%%%%%%%%%
\section{Scalar field perturbation under diffeomorphism}

%%%%%%%%%%%%%%%%%%%%%%%%%%%%%%%%

%%%%%%%%%%%%%%%%%%%%%%%%%%%%%%%%
\section{Perturbation of Arnowitt--Deser--Misner Hamiltonian formalism}

%%%%%%%%%%%%%%%%%%%%%%%%%%%%%%%%

The well known Arnowitt--Deser--Misner's Hamiltonian action for gravitation is 
\cite[ch.4.2.2]{Kiefer2012}
\begin{align}
S &= \int\dd t\,\dd x^3\,\cbr{
\mfrakp^{ij}\dot{h}_{ij} + \mfrakP\dot{N} + \mfrakP^i \dot{N}_i
-N\mfrakH^\perp - N_i\mfrakH^i - \mfrakP V - \mfrakP^i V_i}, \\
\mfrakH^\perp &= \frac{\varkappa}{\sqrt{\mfrakh}}\rbr{
h_{ik}h_{jl} + h_{il}h_{kj} - h_{ij}h_{kl}} \mfrakp^{ij}\mfrakp^{kl}
- \frac{\sqrt{\mfrakh}}{2\varkappa} {}^{(3)}R, \\
\mfrakH^i &= -2 \mfrakp^{ij}{}_{|j},
\end{align}
where $\cbr{V, V_i}$ are velocity of $N$ and $N_i$ and play the role of 
Lagrange multipliers. Note that $\cbr{N, N_i, h_{ij}; \mfrakP, \mfrakP^i, 
\mfrakp^{ij}}$ are not the unique choice of canonical variables for General 
Relativity in Hamiltonian formalism; instead, they are a special parametrisation 
of the phase space. One can also choose the components of the original 
four-metric and their conjugate momenta $\cbr{g_{\mu\nu}, \mfrakp^{\mu\nu}}$ as 
canonical variables, as \citeauthor{Dirac1958} has done \cite{Dirac1958}. The 
two approaches are different in some subtle aspects; see \cite{Kiriushcheva2008} 
for a comparison.

Gauge transformations in the Arnowitt--Deser--Misner variables are generated by 
\cite{Castellani1982} 
\begin{align}
G &= -\int\dd^3 x\,\Big\{ \sbr{
	\xi^\perp \rbr{
		\mfrakH^\perp + N_{|i} \mfrakP^i + \rbr{N\mfrakP^i}_{|i} + 
			\rbr{N^i \mfrakP}_{|i}} +
	\dot{\xi}_\perp \mfrakP}
	\nonumber \\ &\qquad\qquad\ + \sbr{
	\xi_i \rbr{
		\mfrakH^i + N_j{}_{|i} \mfrakP^j + \rbr{N_j \mfrakP^i}^{|j} + N^{|i} 
\mfrakP} +
	\dot{\xi}_i \mfrakP^i} \Big\},
\end{align}
and the infinitesimal gauge transformation of $N$ is
\begin{align}
\dva N &= \sbr{N, G}_\text{P} =
\xi^\perp_{|j}N^j - \dot{\xi}^\perp - \xi^i N_{|i}, \\
\dva N_i &= %\sbr{N_i, G}_\text{P} =
- \xi^\perp N_{|i} + \xi^\perp_{|i} N
- \xi_j N_{i}{}^{|j} + \xi_i{}^{|j} N_j - \dot{\xi}_i,
\end{align}
which can be found in \cite{Kiriushcheva2008}. The transformation for 
$g_{ij}$ reads
\begin{align}
\dva g_{ij} &= -\xi^\perp \frac{2\varkappa}{N\sqrt{\mfrakh}}
\rbr{h_{ik}h_{jl}+h_{il}h_{kj}-h_{ij}h_{kl}}\mfrakp^{kl}
- \xi_{i|j} - \xi_{j|i},
\end{align}
which can also be found but some paraphrases are needed. Transformations for 
the momenta have to be worked out as
\begin{align}
\dva \mfrakP &= ,\\
\dva \mfrakP^i &= ,\\
\dva \mfrakp^{ij} &= .\\
\end{align}





\printbibliography

\end{document}
