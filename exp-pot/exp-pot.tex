\documentclass[a4paper,11pt]{article}
%\documentclass[a4paper,11pt]{scrartcl}



\input{../preambles/preamble}
\input{../preambles/unicode}

\setmainlanguage{english}
\setotherlanguages{german,greek,russian}

\input{../preambles/math-single}
\input{../preambles/math-brac}
\input{../preambles/math-thm}
\input{../preambles/phys-chem}

\setromanfont[Mapping=tex-text]{Linux Libertine O}
% \setsansfont[Mapping=tex-text]{DejaVu Sans}
% \setmonofont[Mapping=tex-text]{DejaVu Sans Mono}

\usepackage[%style=authoryear-icomp,
			backend=biber]{biblatex}
\addbibresource{./exp-pot.bib}

\title{Non-relativistic Particle in an Exponential Potential}
\author{Yi-Fan Wang (王\ 一帆)}
%\date{}

\begin{document}
\maketitle

Consider the one-dimensional motion of a non-relativistic particle in an 
exponential potential, the motion of which can be described by the Lagrangian 
action
\begin{align}
S \coloneqq \int \dd t\,\cbr{\frac{m}{2}\dot{x}^2 - V\ee^{g x}},
\end{align}
where $g$ and $V$ are real quantities. One sees that when $V > 0$ ($< 0$), the 
potential is bounded below (above), and the second case is potentially 
problematic.

%%%%%%%%%%%%%%%%%%%%%%%%%%%%%%%%
\section{Canonical formalism}

%%%%%%%%%%%%%%%%%%%%%%%%%%%%%%%%

The canonical Hamiltonian of the particle reads
\begin{align}
H = \frac{p^2}{2m} + V\ee^{g x}.
\end{align}

%%%%%%%%%%%%%%%%%%%%%%%%%%%%%%%%
\section{Canonical quantisation}

%%%%%%%%%%%%%%%%%%%%%%%%%%%%%%%%

Using the Laplace--Beltrami operator, the Hamiltonian ``operator'' reads
\begin{align}
\widehat{H} = -\frac{\phs^2}{2m}\partial_x^2 + V\ee^{g x}.
\end{align}
Note that the domain of the unbounded operator has not been specified; hence 
comes the quotation marks. In \cite[ch.\ 4]{Gitman2012}, it was suggested 
that one could use \emph{operation} instead of ``operator'' to distinguish the 
case, where only the action of an operator is described, whereas the domain is 
not.

%%%%%%%%%%%%%%%%
\subsection{Spectrum and generalised eigenfunctions of the Hamiltonian}

%%%%%%%%%%%%%%%%

The eigenvalue equation of the Hamiltonian, or the time-independent Schrödinger 
equation, reads
\begin{align}
-\frac{\phs^2}{2m}\partial_x^2 \rfun{\psi}{x} + V\ee^{g x} \rfun{\psi}{x} =
E \rfun{\psi}{x}.
\end{align}
Defining
\begin{align}
v \coloneqq \frac{\sqrt{8m\vbr{E}}}{g\phs},
\end{align}
and transforming the coordinate
\begin{align}
\xi \coloneqq \frac{\sqrt{8m\vbr{V}\ee^{g x}}}{g\phs}
\end{align}
yield the Hamiltonian ``operator'' in terms of dimensionless variables
\begin{align}
\widehat{H} = \frac{g^2\phs^2}{8m}\rbr{-\xi^2\partial_\xi^2 - \xi\partial_\xi + 
\mscrv \xi^2},\qquad \mscrv \coloneqq \sgn V
\end{align}
and the corresponding eigenvalue equation in the standard Besselian form
\begin{align}
\xi^2 \rfun{\psi''}{\xi} + \xi^1 \rfun{\psi'}{\xi} +
\rbr{-\mscrv \xi^2 + \mscre v^2} \rfun{\psi}{\xi} = 0,\qquad
\mscre \coloneqq \sgn E.
\end{align}
Transforming
\begin{align}
\ee^y \coloneqq \xi = \frac{\sqrt{8m\vbr{V}\ee^{g x}}}{g\phs}
\end{align}
yields the Hamiltonian ``operator'' in terms of an alternative dimensionless 
form
\begin{align}
\widehat{H} = \frac{g^2\phs^2}{8m}\rbr{-\partial_y^2 + \mscrv \ee^{2y}}.
\end{align}



%%%%%%%%%%%%%%%%
\subsection{}

%%%%%%%%%%%%%%%%

\printbibliography

\end{document}
