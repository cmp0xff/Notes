\documentclass[a4paper,11pt]{article}
%\documentclass[a4paper,11pt]{scrartcl}



% Fonts and languages

% Multilingual support
%\usepackage{polyglossia}

% more symbols
\usepackage{textcomp}

% AMS--related
\usepackage{amsmath,amssymb}

% ':=' as \coloneqq
\usepackage{mathtools}
% Physical bras and kets
\usepackage{braket}
% SI units
\usepackage{siunitx}
\sisetup{separate-uncertainty}

\usepackage{graphicx}
%\usepackage[colorinlistoftodos]{todonotes}


% Boxed equations. NEED TO BE LOADED BEFORE unicode-math!
\usepackage{empheq}
% Theorems
\usepackage{amsthm}

% Chemical elements
%\usepackage[version=4]{mhchem}


%\usepackage{fancybox}

\usepackage{enumitem} % \begin{enumerate}[label=\Alph*]


% Other formats

% Labelling equations according to sections
\numberwithin{equation}{section}


% Bibliography in the main text!!!

%\usepackage[
			%style=alphabetic,
%			backend=biber]{biblatex}
%\usepackage{hyperref}
\usepackage{nameref}

% Cross-references
% The \newtheorem commands have to come after the loading of {cleveref}.
% Additionally, the cleverref package has to be loaded after ntheorem or
% amsthm. cleverref has to be loaded after hyperref!
\usepackage{cleveref}
%\usepackage{nameref}%,thmtools}

% Plot
\usepackage{tikz}
\usetikzlibrary{decorations.pathmorphing}
\usetikzlibrary{calc}
% save compiled tikz plots; enable --shell-escape
%\usetikzlibrary{external}
%\tikzexternalize[prefix=./tikz/]

% Fontspec fot XeLaTeX
\usepackage{fontspec}
	% Unicode fonts
	\setmainfont{CMU Serif}
	\setsansfont{CMU Sans Serif}
	\setmonofont{CMU Typewriter Text}
	% declare a command \doulos to load the Doulos SIL font
	%\newfontfamily\brill{Brill}
	% now create a \textIPA{} command
	%\DeclareTextFontCommand{\textIPA}{\brill}
\usepackage{amsfonts}
\usepackage{unicode-math}
\usepackage{unicode-math}
	\setmathfont{Latin Modern Math} % default
	%\setmathfont[range=\mathalpha]{Asana Math}
	\setmathfont{Asana Math}[range={\mathbin}] %\mathord
	\setmathfont{STIX Math}[range={"02609}] % ☉
	\setmathfont{XITS Math}[range={"1D4B6-"1D4CF}] % Script, Latin, lowercase
	\setmathfont{Latin Modern Math}[range={"1D608-"1D63B}, sans-style=italic]
	\setmathfont{Latin Modern Math}[range={
		"00391-"003A9,
		"003B1-"003F5, 
		"1D6A8-"1D6E1},	% Bold Greek
		sans-style=upright]
	%\setmathfont{⟨font name⟩}[range=⟨unicode range⟩,⟨font features⟩]

\input{../preambles/unicode}

\setmainlanguage{english}
\setotherlanguages{german,greek,russian}

\input{../preambles/math-single}
\input{../preambles/math-brac}
\input{../preambles/math-thm}
\input{../preambles/phys-chem}

\setromanfont[Mapping=tex-text]{Linux Libertine O}
% \setsansfont[Mapping=tex-text]{DejaVu Sans}
% \setmonofont[Mapping=tex-text]{DejaVu Sans Mono}

\usepackage[%style=authoryear-icomp,
			backend=biber]{biblatex}
\addbibresource{./exp-pot.bib}

\title{Non-relativistic Particle in an Exponential Potential}
\author{Yi-Fan Wang (王\ 一帆)}
%\date{}

\begin{document}
\maketitle

Consider the one-dimensional motion of a non-relativistic particle in an 
exponential potential, the motion of which can be described by the Lagrangian 
action
\begin{align}
S \coloneqq \int \dd t\,\cbr{\frac{m}{2}\dot{x}^2 - V\ee^{g x}},
\end{align}
where $g$ and $V$ are real quantities. One sees that when $V > 0$ ($< 0$), the 
potential is bounded below (above), and the second case is potentially 
problematic.

%%%%%%%%%%%%%%%%%%%%%%%%%%%%%%%%
\section{Canonical formalism}

%%%%%%%%%%%%%%%%%%%%%%%%%%%%%%%%

The canonical Hamiltonian of the particle reads
\begin{align}
H = \frac{p^2}{2m} + V\ee^{g x}.
\end{align}

%%%%%%%%%%%%%%%%%%%%%%%%%%%%%%%%
\section{Canonical quantisation}

%%%%%%%%%%%%%%%%%%%%%%%%%%%%%%%%

Using the Laplace--Beltrami operator, the Hamiltonian ``operator'' reads
\begin{align}
\widehat{H} = -\frac{\phs^2}{2m}\partial_x^2 + V\ee^{g x}.
\end{align}
Note that the domain of the unbounded operator has not been specified; hence 
comes the quotation marks. In \cite[ch.\ 4]{Gitman2012}, it was suggested 
that one could use \emph{operation} instead of ``operator'' to distinguish the 
case, where only the action of an operator is described, whereas the domain is 
not.

%%%%%%%%%%%%%%%%
\subsection{Spectrum and generalised eigenfunctions of the Hamiltonian}

%%%%%%%%%%%%%%%%

The eigenvalue equation of the Hamiltonian, or the time-independent Schrödinger 
equation, reads
\begin{align}
-\frac{\phs^2}{2m}\partial_x^2 \rfun{\psi}{x} + V\ee^{g x} \rfun{\psi}{x} =
E \rfun{\psi}{x}.
\end{align}
Defining
\begin{align}
v \coloneqq \frac{\sqrt{8m\vbr{E}}}{g\phs},
\end{align}
and transforming the coordinate
\begin{align}
\xi \coloneqq \frac{\sqrt{8m\vbr{V}\ee^{g x}}}{g\phs}
\end{align}
yield the Hamiltonian ``operator'' in terms of dimensionless variables
\begin{align}
\widehat{H} = \frac{g^2\phs^2}{8m}\rbr{-\xi^2\partial_\xi^2 - \xi\partial_\xi + 
\mscrv \xi^2},\qquad \mscrv \coloneqq \sgn V
\end{align}
and the corresponding eigenvalue equation in the standard Besselian form
\begin{align}
\xi^2 \rfun{\psi''}{\xi} + \xi^1 \rfun{\psi'}{\xi} +
\rbr{-\mscrv \xi^2 + \mscre v^2} \rfun{\psi}{\xi} = 0,\qquad
\mscre \coloneqq \sgn E.
\end{align}
Transforming
\begin{align}
\ee^y \coloneqq \xi = \frac{\sqrt{8m\vbr{V}\ee^{g x}}}{g\phs}
\end{align}
yields the Hamiltonian ``operator'' in terms of an alternative dimensionless 
form
\begin{align}
\widehat{H} = \frac{g^2\phs^2}{8m}\rbr{-\partial_y^2 + \mscrv \ee^{2y}}.
\end{align}



%%%%%%%%%%%%%%%%
\subsection{}

%%%%%%%%%%%%%%%%

\printbibliography

\end{document}
