% Math symbols and user-defined extensions


% some unicode characters
% ≙ for equal with hat


% Mathematical constants
\newcommand{\ii}{{\Bbbi}}
\newcommand{\ee}{{\Bbbe}}
\newcommand{\pp}{{\Bbbpi}}

% Bracket-like
\newcommand{\rbr}[1]{{\left(#1\right)}}
\newcommand{\sbr}[1]{{\left[#1\right]}}
\newcommand{\cbr}[1]{{\left\{#1\right\}}}
\newcommand{\abr}[1]{{\left<#1\right>}}
\newcommand{\vbr}[1]{{\left|#1\right|}}
\newcommand{\dvbr}[1]{{\left\|#1\right\|}}
\newcommand{\fat}[2]{{\left.#1\right|_{#2}}}
% Functions; note the space between the name and the bracket!
\newcommand{\rfun}[2]{{#1}\mathopen{}\left(#2\right)\mathclose{}}
\newcommand{\sfun}[2]{{#1}\mathopen{}\left[#2\right]\mathclose{}}
\newcommand{\cfun}[2]{{#1}\mathopen{}\left\{#2\right\}\mathclose{}}
\newcommand{\afun}[2]{{#1}\mathopen{}\left<#2\right>\mathclose{}}
\newcommand{\vfun}[2]{{#1}\mathopen{}\left|#2\right|\mathclose{}}
% Differentials
\newcommand{\DD}{\BbbD}
\newcommand{\dd}{\Bbbd}
\newcommand{\dva}{\mupdelta} % no better way?!
% Fraction-like
\newcommand{\frde}[2]{{\frac{\dd{#1}}{\dd{#2}}}}
\newcommand{\frDe}[2]{{\frac{\DD{#1}}{\DD{#2}}}}
\newcommand{\frpa}[2]{{\frac{\partial{#1}}{\partial{#2}}}}
\newcommand{\frdva}[2]{{\frac{\dva{#1}}{\dva{#2}}}}
% Equal marks
\newcommand{\eeq}{{\overset{!}{=}}}
\newcommand{\lls}{{\overset{!}{<}}}
\newcommand{\ggt}{{\overset{!}{>}}}
\newcommand{\lle}{{\overset{!}{\le}}}
\newcommand{\gge}{{\overset{!}{\ge}}}
% overline-like marks
\newcommand{\ol}[1]{{\overline{{#1}}}}
\newcommand{\ul}[1]{{\underline{{#1}}}}
\newcommand{\tld}[1]{{\widetilde{{#1}}}}
\newcommand{\ora}[1]{{\overrightarrow{#1}}}
\newcommand{\ola}[1]{{\overleftarrow{#1}}}
\newcommand{\td}[1]{{\widetilde{#1}}}
\newcommand{\what}[1]{{\widehat{#1}}}
%\newcommand{\prm}{{\symbol{"2032}}}

% Math operators
% Why does \DeclareMathOperator not work?
\DeclareMathOperator{\sgn}{sgn}
\DeclareMathOperator{\grad}{grad}
\DeclareMathOperator{\curl}{curl}
\DeclareMathOperator{\rot}{rot}
\DeclareMathOperator{\opdiv}{div}
\DeclareMathOperator{\opdeg}{deg}

\DeclareMathOperator{\sech}{sech}
\DeclareMathOperator{\csch}{csch}

\DeclareMathOperator{\diag}{diag}
\DeclareMathOperator{\tr}{tr}

\DeclareMathOperator{\ad}{ad}

\DeclareMathOperator{\expi}{expi}

% Group and Algebras
\newcommand{\SO}{\mathsf{SO}\,}
\newcommand{\SU}{\mathsf{SU}\,}
\newcommand{\so}{\mathfrak{so}\,}
\newcommand{\su}{\mathfrak{su}\,}


% amsthm
\newcommand{\thistheoremname}{} % for generic stuff
% definition of new styles
\newtheoremstyle{varplain}% name
	{}{}%      Spaces above and below, empty = `usual value'
	{\itshape}% Body font
	{}%         Indent amount (empty = no indent, \parindent = para indent)
	{\bfseries}% Thm head font
	{}%        Punctuation after thm head
	{\newline}% Space after thm head: \newline = linebreak
	{{\normalfont\thmnumber{(#2)}}\thmname{ #1}{\normalfont\thmnote{ (#3)}}}
	%         Thm head spec
\newtheoremstyle{vardefinition}% name
	{}{}%      Spaces above and below, empty = `usual value'
	{\upshape}% Body font
	{}%         Indent amount (empty = no indent, \parindent = para indent)
	{\bfseries}% Def head font
	{}%        Punctuation after thm head
	{\newline}% Space after thm head: \newline = linebreak
	{{\normalfont\thmnumber{(#2)}}\thmname{ #1}{\normalfont\thmnote{ (#3)}}}
	%         Thm head spec
\newtheoremstyle{varremark}% name
	{}{}%      Spaces above and below, empty = `usual value'
	{\upshape}% Body font
	{}%         Indent amount (empty = no indent, \parindent = para indent)
	{\itshape}% Rem head font
	{}%        Punctuation after thm head
	{\newline}% Space after thm head: \newline = linebreak
	{{\normalfont\upshape\thmnumber{(#2)}}\thmname{ #1}{\normalfont\thmnote{ (#3)}}}
	%         Thm head spec
% Plain style
\theoremstyle{plain}% default
\newtheorem{thm}{Theorem}[section]
\newtheorem{lem}[thm]{Lemma}
\newtheorem{prop}[thm]{Proposition}
% Definition style
\theoremstyle{definition}
\newtheorem{defn}{Definition}[section]
\newtheorem{exmp}{Example}[section]
\newtheorem{ppty}{Property}[section]
% Remark style
\theoremstyle{remark}
\newtheorem*{rem}{Remark}
% variant plain style
\theoremstyle{varplain}
% for specifying a named theorem with numbering
\newtheorem{genericthm}{\thistheoremname}[section]
\newenvironment{namedthm}[1]
	{\renewcommand{\thistheoremname}{#1}%
		\begin{genericthm}}
	{\end{genericthm}}
% for specifying a named theorem without numbering
\newtheorem*{genericthm*}{\thistheoremname}
\newenvironment{namedthm*}[1]
	{\renewcommand{\thistheoremname}{#1}%
		\begin{genericthm*}}
	{\end{genericthm*}}
\newtheorem{unamedthm}[genericthm]{Theorem}
% variant definition style
\theoremstyle{vardefinition}
% for specifying a named definition with numbering
\newtheorem{genericdef}[genericthm]{\thistheoremname}
\newenvironment{nameddef}[1]
	{\renewcommand{\thistheoremname}{#1}%
		\begin{genericdef}}
	{\end{genericdef}}
% for specifying a named definition without numbering
\newtheorem*{genericdef*}{\thistheoremname}
\newenvironment{nameddef*}[1]
	{\renewcommand{\thistheoremname}{#1}%
		\begin{genericdef*}}
	{\end{genericdef*}}
\newtheorem{unameddef}[genericthm]{Definition}
% variant remark style
\theoremstyle{varremark}
% for specifying a named  with numbering
\newtheorem{genericrem}[genericthm]{\thistheoremname}
\newenvironment{namedrem}[1]
	{\renewcommand{\thistheoremname}{#1}%
		\begin{genericrem}}
	{\end{genericrem}}
% for specifying a name without numbering
\newtheorem*{genericrem*}{\thistheoremname}
\newenvironment{namedrem*}[1]
	{\renewcommand{\thistheoremname}{#1}%
		\begin{genericrem*}}
	{\end{genericrem*}}
\newtheorem{unamedrem}[genericthm]{Remark}


% cleveref
\crefname{lem}{lemma}{lemmas}
\Crefname{lem}{Lemma}{Lemmas}
\crefname{thm}{theorem}{theorems}
\Crefname{thm}{Theorem}{Theorems}
\crefname{defn}{definition}{definitions}
\Crefname{defn}{Definition}{Definitions}
\crefname{exmp}{example}{examples}
\Crefname{exmp}{Example}{Examples}
\crefname{namedthm}{}{}
\Crefname{namedthm}{}{}